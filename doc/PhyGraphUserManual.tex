\documentclass[11pt]{article}
\usepackage{longtable}
\usepackage{color}
\usepackage{tabu}
\usepackage{setspace}
\usepackage{pdflscape}
\usepackage{graphicx}
%\usepackage{subfigure}
\usepackage{caption}
\usepackage{subcaption}
\usepackage{natbib}
\usepackage{fullpage}
%\bibliographystyle{plain}
\bibliographystyle{cbe}
\usepackage{algorithmic}
\usepackage[vlined,ruled]{algorithm2e}
\usepackage{amsmath}
\usepackage{amsfonts}
\usepackage{amssymb}
\usepackage{url}
\usepackage[dvipsnames]{xcolor}
\usepackage[colorlinks=true, linkcolor=blue, citecolor=DarkOrchid, urlcolor=TealBlue ]{hyperref}
\usepackage[nottoc,numbib]{tocbibind}

\let \oriAlgorithm=\algorithm%
\renewcommand{\algorithm}[2]{%
	\oriAlgorithm%
	\SetAlgoLined%
	\DontPrintSemicolon%
	\SetKwComment{Comment}{$\triangleright\ $}{}%
	\TitleOfAlgo{ #1 }%
	\caption{ #2 }%
}%
\newcommand*{\Set}[2]{ #1 $\gets$ #2 \;}
\newcommand*{\SetC}[3]{ #1 $\gets$ #2 \Comment*[r]{#3}}
\newcommand*{\ForC}[3]{\For(\Comment*[f]{#2}){#1}{#3}}

\setcounter{tocdepth}{2} 

\begin{document}
	%\firstpage{1}
	
	\title{PhylogeneticGraph\\User Manual\\Version 0.1}
	%\subtitle{Program Documentation\\Version 0.1}
	\author{Ward C. Wheeler\\
		Division of Invertebrate Zoology,\\ American Museum of Natural History,\\ 200 Central Park West, New York, NY, 10024, USA;\\wheeler@amnh.org}
	
	
	\maketitle
%	\begin{center}
%		Running Title: {PhyGraph Version 0.1}
%	\end{center}
	\newpage
	
	\tableofcontents
	\newpage
	\section{Introduction}
	This is the initial version of documentation for the program PhylogeneticGraph (\texttt{PhyG}).  This program is designed to produce phylogenetic graphs from input data and graphs via heuristic searching of general phylogenetic graph space.
	
	 \texttt{PhyG} is a sucessor program to \texttt{POY} \citep{POY2,POY3,POY4,Varonetal2010,POY5, Wheeleretal2015} \url{https://github.com/wardwheeler/POY5}, containing much of its functionality but extended to broader classes of input data and phylogenetic graphs.
	The phylogenetic graph inputs and outputs of \texttt{PhyG} include not only trees, but other forms including forests and both
	soft and hard-wired networks.
	
	
	
	\section{Overview of program use}
	At present, \texttt{PhyG} is operated solely via command-line interface. 
	
	Commands are entered via a scriptfile that contains commands, which may themselves contain references to other script files.
	These commands specify input files, output files and formats, graph type and search parameters. 
	
	\bigskip
	\texttt{phyg script.pg}
	
	
	\section{QuickStart}
		\subsection{Requirements}
		\subsection{Obtaining and Installing \texttt{PhyG}}
			All source code, precompiled binaries, test data, and documentation are available from \url{https://githib.com/wardwheeler/PhyGraph}.

	\section{Input Data Formats}
		Any character names in input files are (for now) ignored and internal names are created
		by appending the character number in its file to the filename as in "fileName:0".
		Qualitative data, and prealigned data include their index in their input files and unaligned 
		data are treated as a single character.
		\subsection{fasta}
			Single character sequence input \citep{PearsonandLipman1988}.
		\subsection{fastc}
			Multicharacter sequence input.  \citep{WheelerandWashburn2019}.
		\subsection{\texttt{TNT}}
		The TNT \citep{Goloboffetal2008} format is accepted here for specification of qualitative, measurement, and prealigned molecular 
		sequence data.  \texttt{PhyG} does not parse all the diversity of options that can be specified in \texttt{TNT} input files.
		Do not support everything, interleave yes, cc code and costs yes, but one set of commands per line.
		Costs A$>$B A$/$B syntax no spaces. Ambiguities not allowed.  Must specify all transformations manually.
		')' sets to non-additive, if wantadditive then need to reset to additive after\\
		character state designations single characters\\
		continuous and other multicharacter states--can be read, abinguitess or ranges (unlike tnt for continuous) are 
		in squarebraskets with period as in [X.Y]\\
		- ans ? are always missing/inapplicable.  If DNA are not coded ACGT- => 01234 but left as lettes,
		then to include gaps as a 5th state, just include another character, such the letter 'O' that is not an IUPAC ambiguity code for DNA (or amino acids for that matter) as an additional state.  This can be used for matrix/Sankoff matrices as well.
		Amino acid sequences would be processed in the same way.  Including gaps as information requires
		an extra state (such as letter 'O').\\ 
		multi-character state designations (letters, numbers, etc) must be in their own ``block'' with spaces 
		between them.\\
		Continuous characters must be numbers only (float fine) and declares by cccode command as 
		``additive'', otherwise the number will be treated as non-additive character states. \\
		nonAdd polymorphisms  are [X.Y]--unless  additive '-' for range\\
		can't have '.' in multichar state (or single for that matter)\\
		Inherent ordering of DNA and AMino Acid codes is alphabetical (e.g. A,C,G,T.`-')
	
	\section{Input Graph Formats}
	Graphs may be input in the graphviz ``dot'' format \url{https://graphviz.org/}, Newick (as interpreted by Gary Olsen; \url{https://evolution.genetics.washington.edu/phylip/newick_doc.html}), Enhanced Newick \cite{Cardonaetal2008},
	and Forest Enhanced Newick (defined by \citealp{WheelerPhyloSuperGraphs}) formats.
	
	Quickly, Forest Enhanced Newick (FEN) is a format based on Enhanced Newick (ENewick) for forests of components, each
	of which is represented by an ENewick string.  The ENewick components are surrounded by `$<$' and '$>$'.
	As in $<$(A, (B,C)); (D,(E,F));$>$.  Groups may be shared among ENewick components.
	
	
	\section{Output Graph Formats}
	Graph outputs can be in either Graphviz `dot' or FEN formats.  Dot files can be visualized in a variety of ways 
	using Graphviz (e.g. dot, neanto, twopi) into pdf, jpg and a large variety of other formats. FEN outputs of 
	single trees (ie forest with a single component) are rendered as enewick.  Newick files can be visualized in a 
	large number of programs (e.g. FigTree; \url{http://tree.bio.ed.ac.uk/software/figtree/}, Dendroscope; 
	\url{https://uni-tuebingen.de/fakultaeten/mathematisch-naturwissenschaftliche-fakultaet/fachbereiche/informatik/lehrstuehle/algorithms-in-bioinformatics/software/}). 	
	When FEN/Enewick files are output, leaf vertices are modified if they have indegree $>$ 1, creating a new node as parent to that leaf
	and redirecting the leaf's in-edges to that new node with a single edge connecting the new node to the leaf.  
	\bigskip
	\\Example dot command line: \\dot -Tpdf myDotFile.dot $>$ myDotFile.pdf\\
	
	Mulitple ``dot'' graphs can be output in a single file.  To create pdf and 
	other formats the commandline would be (these files are named and numbered automatically):\\\\dot -Tpdf -O myDotFile.dot\\
	
	For some reason on OSX the `pdf' option does not seem to work.  You can use `-Tps2' and that will generate a postscript file ($>$ blah.ps) that Preview can read and convert to pdf.
	
	\section{Command options}
	There are only a few program options that require specification.  There are defaults for all but input graphs.  
	Parameters are given with options in a range `a to b' (a-b) with any value in the interval, or alternates `a or b' (a|b). File options require a valid filename.
	For input graphs, wildcards are allowed (ie `*' and `?').  All commands are followed by a colon `:' before the option with no spaces.  Capitalization (for commands, but not filenames) is ignored.  Commands can be in any order (or entered from a file as stdin `$<$ filename).
	
	

	
	The program requires at least one input graph file and at least two input graphs (they could be in the same file).

	
	\section{Program Use}
	The program is invoked from the command-line as in:\\
	PhyGraph commandFile\\
	
	\section{Commands}
		\subsection{Command Structure}
		Commands and arguments are case insensitive with the exception of filename specifications, which are also 
		always in double quotes (\texttt{"fileName"}).  Commands are specified with mandatory parentheses \texttt{`()'} which may be empty or
		contain command arguments. Arguments may be preceded or followed by options separated by a colon \texttt{`:'}.  Command arguments are not 
		order dependent.
		
		\bigskip
		\texttt{command(argument, option:argument, option:argument...)}
		\subsection{Command order and processing}
	
	
	
	\subsection{Build}
		\subsubsection{Syntax}
		\texttt{build(arg0, arg1:option, ...)}
		\subsubsection{Description}
		Builds initial graphs.  Most options (with the exception of the $O(n^3)$ distance based \texttt{dWag} option and character based \texttt{bWag})
		are $O(n^2)$.  Builds may be based on character methods or those based on distance methods.  Distance methods are considerably faster 
		(lower constant factor), but approximate in terms of character-based methods.  Refinement (branch swapping) can be specified
		for distance builds.  Given the large time burden, distance refinement is usually not time effective \citep{Wheeler2021}.
		\subsubsection{Arguments}
		\texttt{distance} Causes a pairwise distance matrix to be calculated ($O(n^2)$) and used as a basis for tree construction.
		
		\noindent \texttt{dWag} Performs distance Wagner build as in \citep{Farris1972} choosing the `best' taxon to add at each step, 
		yielding a single tree.  Time complexity of $O(n^3)$.
		
		\smallskip
		\noindent \texttt{rWag} Performs randoms addition sequence distance Wagner builds, yielding a multiple trees determined 
		by argument \texttt{replicates:m}.  Time complexity of $O(m \times n^2)$.
		
		\smallskip		
		\noindent \texttt{nj} Performs Neighbor-Joining build \citep{Saitou1987}, yielding a single tree.  Time complexity of $O(n^3)$.
		
		\smallskip		
		\noindent \texttt{wpgma} Performs Weighted Pair Group Method with Arithmetic Mean build \citep{SokalandMichener1958}, 
		yielding a single tree.  Time complexity of $O(n^2)$.
		
		\smallskip		
		\noindent \texttt{replicates:n}  Applies only to rWag.  Specifies the number of random addition sequences performed. 
		
		\smallskip		
		\noindent \texttt{best:n} Applies only to rWag.  Specifies the number of trees retained after rWag builds, selecting 
		the best trees in terms of distance cost.  The options can be used to reduce
		the number of trees retained for refinement or returned for further analysis.  
		
		\smallskip		
		\noindent \texttt{none} No refinement (OTU,SPR,TBR) is performed after distance builds.
		
		\smallskip		
		\noindent \texttt{otu} Specifies that OTU refinement \citep{Wheeler2021} is performed after distance builds.
		
		\smallskip		
		\noindent \texttt{spr} Specifies that SPR refinement \citep{Dayhoff1969} is performed after distance builds.

		\smallskip		
		\noindent \texttt{tbr} Specifies that TBR refinement \citep{Farris1988, swofford1990a} is performed after distance builds.

		\subsubsection{Defaults}
			\texttt{build(distance, dwag)}
		\subsubsection{Examples}
			\begin{itemize}
				\item{\texttt{build(distance, dWag, nj, wpgma)}\\Builds a single `best' addition sequence distance Wagner build, a Neighbor-Joining tree, and a WPGMA tree.}
				\item{\texttt{build(distance, rWag, replicates:1000, best:10)}\\Builds 1000 random addition sequence  Wagner builds and returns  
				10 of lowest distance cost.}
				\item{\texttt{build(distance, rWag, WPGMA, replicates:100, best:10, OTU)} \\Builds 100 random addition sequence  Wagner builds, a WPGMA tree, performs OTU swapping on the WPGMA and 10 of the lowest cost random addition sequence Wagner trees. 
			10 of lowest distance cost.}
			\end{itemize}
	\subsection{Read}
		\subsubsection{Syntax}
				\texttt{read(option:"filename", option:"filename", ...)}
			
		\subsubsection{Description}
		Causes various file-based information to input.  \texttt{read} commands must contain an input file.
		Options prepended to the filename with a colon (`:') modify the processing of the input file. If no options are specified, 
		the program will attempt to recognize the file type and parse appropriately.  Wildcard characters (`*', `?') can be used to specify multiple files.
		\subsubsection{Arguments}
			\noindent \texttt{aminoacid:} File contents are parsed as IUPAC coded amino acid data in fasta \citep{PearsonandLipman1988} format. .
		
			\smallskip		
			\noindent \texttt{block:} Specifies that the file contains block information. Each line contains 
			the new block name followed by names of input files to be assigned to that data block.  
			Blocks are initially named as the input file name with ``:0'' appended.  
			In the examples, data from files ``b'' and ``c'' will be assigned to block ``a''.  There can be no spaces in file or block names.
			\begin{quote}
				\texttt{"a" "b:0" "c:0"}\\
			\end{quote}
			
			\smallskip		
			\noindent \texttt{custom\_alphabet:} Ensures that file contents are parsed in fastc \citep{WheelerandWashburn2019} format. This is used for multi-character sequences such as gene synteny, developmental,  or linguistic  data.
			
			\smallskip		
			\noindent \texttt{dot:} Specifies that the file contains a graph in `dot' format for use with graph rendering software such as GraphViz (\url{https://en.wikipedia.org/wiki/Graphviz}).
			
			\smallskip		
			\noindent \texttt{enewick:} File contains Enhanced Newick format graph of \cite{Cardonaetal2008}.
			
			\smallskip		
			\noindent \texttt{exclude:}  File contains the names of terminal taxa to be excluded from an analysis.  Taxa not include in the list, in input files, will be included in analysis. 
			
			\smallskip		
			\noindent \texttt{fasta:} Ensures that file contents are parsed in fasta \citep{PearsonandLipman1988} format. This is used for single character sequences such as binary streams, IUPAC nucleotide and amino acid sequence data.
			
			\smallskip		
			\noindent \texttt{fastc:} Ensures that file contents are parsed in fastc \citep{WheelerandWashburn2019} format. This is used for multi-character sequences such as gene synteny, developmental,  or linguistic  data.
			
			\smallskip		
			\noindent \texttt{fenewick:}  File contains Forest Enhanced Newick format graph of \cite{Wheeler2021a} in \url{https://www.github.com/wardwheeler/euncon}.
			
			\smallskip		
			\noindent \texttt{include:}  File contains the names of terminal taxa to be included in analysis.  It is possible to specify terminals that have no data.  This may be done to diagnose a large tree on partial data.  If there are no data for a leaf taxon, a warning will be printed to stderr.  Taxa not include in the list will be excluded from analysis. 
			
			\smallskip		
			\noindent \texttt{newick:} File contains  Newick format graph as specified by \url{https://evolution.genetics.washington.edu/phylip/newick_doc.html}.
			
			\smallskip		
			\noindent \texttt{nucleotide:} File contents are parsed as IUPAC coded nucleotide data in fasta \citep{PearsonandLipman1988} format. .
			
			\smallskip		
			\noindent \texttt{prealigned:} Specifies that the sequences are prealigned, leaving gap characters
			(``-'') in the sequences and alignment correspondences are not re-examined. Without this option,
			gap characters are filtered out of the sequences after input.
			
			\smallskip		
			\noindent \texttt{rename:} File contains a series of lines, each of which contains at least two strings.  The second (and all following) strings (input taxon names),
			will be replaced by the first string (taxon name) on the line. An example file renaming ``b'' and ``c''  to ``a'' and ``Ozzy'' and ``Harriet'' to ``OH''.
				\begin{quote}
				\texttt{"a" "b" "c"}\\
				\texttt{"OH" "Ozzy" "Harriet"}
			\end{quote}
		
			 The \texttt{rename} function can also be specified as a command:  \texttt{rename("a","b","c")} where ``b'' and ``c'' will be rename to ``a''. 
			 
			\smallskip		
			\noindent \texttt{tcm:} File containing transformation cost matrix to specify varying costs among alphabet elements in a sequence.  The first line of the file 
			contains the alphabet elements separated by space (the dash character representing an insertion/deletion or indel character is not specified but added to the alphabet automatically).
			A square matrix with $n + 1$ ($n$ is the size of alphabet) follows with a row on each line.  An extra column at the and of each row and an additional row are 
			added to specify indel costs to and from alphabet elements.  At present, this matrix must be symmetrical, but not necessarily metric.  Non-metric tcm's can
			yield unexpected results. Transformation costs must be integers.  If real values are desired, a character can be weighted with a floating point value factor. For a sequence with three elements A, B, and C,  and an indel cost of 4 for all insertion deletion transformations, a example tcm would  be:
			\begin{quote}
			A B C\\
			0 1 2 4\\
			1 0 2 4\\
			2 2 0 4\\
		    4 4 4 0\\
	   		 \end{quote}
   		 
   		 	\smallskip		
   		 	\noindent \texttt{terminals:}  File contains the names of terminal taxa to be included in analysis.  It is possible to specify terminals that have no data.  This may be done to diagnose a large tree on partial data.  If there are no data for a leaf taxon, a warning will be printed to stderr.  Taxa not include in the list will be excluded from analysis. 
	   
			\smallskip		
			\noindent \texttt{tnt:} Ensures that file contents are parsed in TNT \citep{Goloboffetal2008} format.
			Not all TNT data commands are supported.  Files must begin with \texttt{xread}, followed by an optional comment in single quotes ('comment'), the number of characters and taxa.  Data follow starting on a new
			line.  Taxon names are followed state data.  Data may be in multiple blocks (interleaved) format.  These
			blocks may consist of a series of single character states  without spaces between them, or multiple (or single) character states (e.g. \texttt{Blue}) with space between the individual codings. Blocks must be of all
			one type (ie. single character codings without spaces, or multi-character separated by spaces).
			The data block must be followed by a single semicolon (``;'') on its own line.
			
			Character settings (i.e. \texttt{cc} commands) follow the data block (and semicolon)
		    beginning on a new line.  Activity (\texttt{[,])}, addiitivity/ordered (\texttt{+}), non-additivity/unordered
		    (\texttt{-}), matrix costs (\texttt{(}) can be specified with scopes (e.g. \texttt{cc + 10 12;}, \texttt{cc (.;
		    costs 0 = 0/1 1 0/2 2 0/3 3 0/4 4 1/2 1 1/3 2 1/4 3 2/3 1 2/4 2 3/4 1;}) including abbreviated scopes (\texttt{cc -.;}). Each character setting statement must be terminated by a semicolon (\texttt{;}).  There may
	        be multiple character setting statements in a single line. 
		    
		    Character settings must be followed by \texttt{proc/;} on its own line. \texttt{PhyG} will not process
		    any file contents that follow  \texttt{proc/;}.
		    
		    Additive/ordered character states must be numbers (integer or floating point).  Ranges for continuous
		    characters are specified with a dash wihtin square brackets (e.g. \texttt{[1-2.1]})
		    
		    State polymorphism are specified in square brackets without spaces for
		    single character states, and will spaces for multi-character
		    
		    \textbf{NB--An important difference between \texttt{TNT} and \texttt{PhyG} is in the 
		    treatment of dashes (\texttt{-}).}  Single character dashes are treated as a state (as in sequence or other 
		    ``gap'') whereas in TNT, by default they are missing data (`?'-- unless DNA input is specified).  Dashes in multi-character states (e.g. \texttt{Blue-ish}) are treated as part of the character state specification.
		    If the user wishes that dashes be treated as missing data (`?'), the file must be edited to reflect this
		    by replacing the dashes that are to be treated as missing data with question marks (`?').
		    
		    Example file:
		    	\begin{quote}
		    	\texttt{xread\\
		    	'An example TNT file' 5 5\\
		    	A 000\\
		    	B a14\\
		    	C b22\\
		    	D ?33\\
		    	E d04\\}
		    	
		    	\texttt{A Blue-ish -\\
		    	B Green-ish OneFish\\
		    	C Rather-Red TwoFish\\
		    	D Almost-Cyan RedFish\\
		    	E Orange-definitely BlueFish\\
		    	;\\
		    	cc .;\\
		    	cc + 2;\\
		    	proc/;\\}
		    \end{quote}
			
		\subsubsection{Defaults}
			\texttt{read("filename")} reads data from filename and attempts to recognize the file type
			and process it accordingly.  The assumed file type is printed to stderr for verification.
		\subsubsection{Examples}
				\begin{itemize}
					\item{\texttt{read(prealigned:"myDnaSequenceFile")}\\ Reads sequence data from ``myDnaSequenceFile'' as prealigned data.}
					\item{\texttt{read(rename:"myRenameFile")}\\ Reads a list of taxa and names to be assigned.}  
			\end{itemize}
%	\subsection{Refine}
%		\subsubsection{Syntax}
%		\subsubsection{Description}
%		\subsubsection{Arguments}
%		\subsubsection{Defaults}
%		\subsubsection{Examples}

	\subsection{Reblock}
	\subsubsection{Syntax}
	\texttt{reblock("newBlockName", "inputFile0", "inputFile1",...)}
	\subsubsection{Description}
	Command to assign input data to ``blocks'' that will follow the same display tree when optimized
	as ``soft-wired '' networks.  By default, each input data file is assigned its own block with the name
	of the input file.  The \texttt{block} command is used to reassign these data to new, combined blocks.
	Spaces are not allowed in block names and will produce \texttt{unrecognized block name} errors.    
	\subsubsection{Arguments}
	The first argument is the block to be created, the remainder are the input data to be 
	assigned to that block. Blocks are initially named as the input file name with ``:0'' appended.  Blocks are reported in \texttt{report(data)} command.
	\subsubsection{Defaults}
	None.
	\subsubsection{Examples}
	\begin{itemize}
		\item{\texttt{reblock("a","b:0","c:0")}\\ Assigns input data from file ``b'' and ``c'' to block ``a''. }
	\end{itemize}
	
	\subsection{Rename}
		\subsubsection{Syntax}
			\texttt{rename("newName", "oldName1", "oldputName",...)}
		\subsubsection{Description}
			Command to rename taxa.  The first string (name) is assigned to taxa with the strings (names) that follow. This can be useful when combining data from
			different sources, such as genbank, or in revising names to reflect taxonomic changes.
		\subsubsection{Arguments}
			Taxon names to assign and be assigned.
		\subsubsection{Defaults}
			None.
		\subsubsection{Examples}
			 \begin{itemize}
			 	\item{\texttt{rename("a","b","c")}\\ Renames ``b'' and ``c'' to ``a''. }
			\end{itemize}
		
	\subsection{Report}
		\subsubsection{Syntax}
				\texttt{report("filename", arg0, arg1,...)}
		\subsubsection{Description}
			Outputs various information to file or stderr.  When a filename is specified, output is sent to that file.  By default, new information printed to a file
			if appended to the file.  The option \texttt{overwrite} overrides the default and rewrites the file.  Many of the report options output in csv output that
			can be easily imported by spreadsheet programs.
		\subsubsection{Arguments}
				\noindent \texttt{crossrefs} Prints source file presence or absence of input data in csv format.  This is useful to check for 
				missing and misnamed taxa.
			
				\smallskip
				\noindent \texttt{data} Prints information about of the input data in csv format.
				
				\smallskip
				\noindent \texttt{diagnosis}  Outputs graph diagnosis information such as vertex and states and edge statistics in csv format. 
			
				\smallskip
				\noindent \texttt{displaytrees}  Outputs graph information for soft-wired networks.  The ``display'' trees are output for each data block. 
				
				\smallskip
				\noindent \texttt{graph} Outputs a graph in format specified by the other arguments in the command.  These are \texttt{dot} for GraphViz graph format,
				\texttt{newick} for Newick, ENewick, or ForestEnewick depending on the graph type, \texttt{ascii} for an ascii rendering.  
				
				\smallskip
				\noindent \texttt{pairdist} Outputs a taxon pairwise distance matrix in csv format.  
				
				\smallskip
				\noindent \texttt{reconcile} Outputs  a single ``reconciled'' graph from all present graphs.  The methods include consensus, supertree, and other supergraph methods as described in \cite{Wheeler2012, Wheeler2021a}.  When \texttt{reconcile} is specified as a command option a series of  other options may be specified to taylor the desired outputs:
					\begin{itemize}
							\item {Method:eun$\mid$cun$\mid$majority$\mid$strict$\mid$Adams\\Default:eun\\This commands specifies the type of output graph.
							EUN is the Edge-Union-Network \citep{MiyagiandWheeler2019}, CUN the Cluster Union Network \citep{Baroni2005},
							majority (with fraction specified by `threshold') specifies that a values between 0 and 100 of either vertices or edges will be retained.  If all inputs are trees with the same leaf set this will be the Majority-Rule Consensus \citep{MargushandMcMorris1981}.
							Strict requires all vertices be present to be included in the final graph.  If all inputs are trees with the same leaf set this will be the Strict Consensus \citep{Schuhandpolhemus1980}. Adams denotes the Adams II consensus \citep{Adams1972}.}
						\item{Compare:Combinable$\mid$identity\\Default:combinable\\Species how group comparisons are to be made.  Either by identical match [(A, (B,C))$\neq$(A,B,C)],
							combinable sensu \cite{Nelson1979} [(A, (B,C)) consistent with (A,B,C)].  This option can be used to specify ``semi-strict'' consensus
							\citep{Bremer1990}.
						}
						\item{Threshold:(0-100)\\Default:0\\Threshold must be an integer between 0 and 100 and specifies the frequency of vertex or edge occurrence in input graphs to be included in the output graph.  Affects the behavior of `eun' and` majority.'
						}
						\item{Connect:True$\mid$False\\Default:True\\Specifies the output graph be connected (single component), potentially creating a root node and new edges labeled with ``0.0''.
						}
						\item{EdgeLabel:True$\mid$False\\Default:True\\Specifies the output graph have edges labeled with their frequency in input graphs.
						}
						\item{VertexLabel:True$\mid$False\\Default:False\\Specifies the output graph have vertices labeled with their subtree leaf set.
						}
						\item{OutFormat:Dot$\mid$FENewick\\Default:Dot\\Specifies the output graph format as either Graphviz `dot' or FEN.
						}
					\end{itemize}				 
				
		\subsubsection{Defaults}
				\texttt{report()} prints input data and output graph diagnosis  to stderr.  Default graph representation is \texttt{dot}.
		\subsubsection{Examples}
				\begin{itemize}
					\item{\texttt{report("outFile", newick, overwrite)}\\ Outputs graphs in newick format to ``outFile'', overwriting any existing information.}
					\item{\texttt{report("outFile", crossrefs)}\\ Outputs presence/absence for taxa in input files.  A `+' is output  if taxa are present in an input data file, 
					and `-' if. File is in csv format.  This can be useful in checking for missing sequence or other data and expected renaming.}
					\item{\texttt{report("outFile", dot, reconcile, method:eun, threshold:51)}\\ Outputs reconciled graph using the Edge-Union-Network method with a minimum edge frequency of 51\% in dot format to ``outFile'', appending to any existing information in ``outFile''.}
				\end{itemize}
			
	\subsection{Run}
		\subsubsection{Syntax}
				\texttt{run("filename")}
		\subsubsection{Description}
			Used to execute a file containing commands.  This can be useful to specify common actions such
			as file inputs and graph construction. 
		\subsubsection{Arguments}
			The only argument is the filename containing commands to be executed.
		\subsubsection{Defaults}
			There are no default \texttt{run} arguments.  
		\subsubsection{Examples}
				\begin{itemize}
				\item{\texttt{run("readFiles.pg")}\\ Executes "readFiles.pg", which may contain multiple input files to be \texttt{read}}
				\item{\texttt{run("searchCommands.pg"}\\ Executes "searchCommands.pg", which may contain commands defining a common search strategy (e.g. \texttt{build}).}
			\end{itemize}
	
	\subsection{Select}
		\subsubsection{Syntax}
			\texttt{set(arg0:option, arg1:option, ...)}
		\subsubsection{Description}
			The command is used to set the method and number of graphs to be saved at any point.
			When multiple graphs are present, the \texttt{select} command will specify which of the graphs to
			keep for further analysis or reporting.
		\subsubsection{Arguments}
			\noindent \texttt{all} Keeps all graphs.
		
			\smallskip		
			\noindent \texttt{best} Keeps graphs of best optimality value.
			
			\smallskip		
			\noindent \texttt{random} Keeps graphs chosen at random.
			
			\smallskip		
			\noindent \texttt{arg:n} Keeps up to \texttt{n} graphs determined by the argument.
		\subsubsection{Defaults}
			\texttt{select()} Keeps all unique graphs of best optimality value.
		\subsubsection{Examples}
				\begin{itemize}
					\item{\texttt{select(random:10)}\\ Keeps up to 10 graphs, selected at random.}
					\item{\texttt{select(best:10)}\\ Keeps up to 10 graphs of best optimality value.}
				\end{itemize}
			
	\subsection{Set}
		\subsubsection{Syntax}
				\texttt{set(arg0:option, arg1:option, ...)}
		\subsubsection{Description}
		The command specifies a number of global values and options include outgroup, graph type, final assignment, and optimality criterion.
		\subsubsection{Arguments}
			\smallskip
			\noindent \texttt{compressResolutions:True|False} Determines whether soft-wired graph resolutions are ``compressed''
			if multiple vertex assignments in alternate display trees are equal.
			
			\smallskip
			\noindent \texttt{criterion:parsimony|pmdl} Sets the optimiality criterion for graph search to be method.  Currently, parsimony and PMDL \citep{WheelerandVaron2022} are supported.
			
			\smallskip
			\noindent \texttt{finalAssignment:DirectOptimization|DO|ImpliedAlignment|IA} Sets the method of determining the ``final''
			sequence states. DirectOptimization (DO) uses the DO method to assign the final states, which is more time consuming 
			than \texttt{ImpliedAlignment}. DO has an additional factor of potentially $O(n^2)$ in sequence length compared to the constant factor for IA due to additional graph traversals.
			
			\smallskip
			\noindent \texttt{graphFactor:nopenalty|W15|PMDL} Sets the network penalty  for a softwired network (\texttt{nopenalty|W15}) or \texttt{PMDL} (for criterion = PMDL).  W15 employs the
			parsimony network penalty of \cite{Wheeler2015}.
			
			\smallskip
			\noindent \texttt{graphType:tree|hardwired|softwired} Sets the phylogenetic graph type to tree, hardwired network, or softwired network.  Forest are allowed 
			by the network options.
			
			\smallskip
			\noindent \texttt{outgroup:"Blah"} Makes the outgroup taxon ``Blah''. 
			
			\smallskip
			\noindent \texttt{rootCost:noRootCost|W15|PMDL} Sets the root cost for a graph.  W15 sets a cost at $\frac{1}{2}$ the cost of `inserting' the root character assignments. 
			The W15 root cost is based on the same rationale as the parsimony network penalty of \cite{Wheeler2015}.
			
		\subsubsection{Defaults} 
		There are no default \texttt{set} arguments.  The default outgroup is the taxon whose name  is lexically first after
		renaming and taxon inclusion/exclusion. For this reason, it is best to specify an outgroup explicitly.
		The default optimality criterion is \texttt{parsimony}, \texttt{CompressResolutions} is set to \texttt{True}, \texttt{FinalAssignment} is set to \texttt{DirectOptimization}, and the default graph type is \texttt{tree}.  The default graphFactor is \texttt{W15} if parsimony is the optimallity criterion and \texttt{PMDL} if PMDL is set as the optimality criterion.  The default rootCost
		is \texttt{noRootCost} if  parsimony is the optimallity criterion and \texttt{PMDL} if PMDL is set as the optimality criterion.
		
		\subsubsection{Examples}
			\begin{itemize}
					\item{\texttt{set(optimality:parsimony)}\\Sets the graph search optimality criterion to parsimony.}
					\item{\texttt{set(compressResolutions:False)}\\Turns off soft-wired graph resolution compression.}
			\end{itemize}
		
%	\subsection{Support}
%		\subsubsection{Syntax}
%		\subsubsection{Description}
%		\subsubsection{Arguments}
%		\subsubsection{Defaults}
%		\subsubsection{Examples}
%		
%	\subsection{Swap} 
%		\subsubsection{Syntax}
%		\subsubsection{Description}
%		\subsubsection{Arguments}
%		\subsubsection{Defaults}
%		\subsubsection{Examples}
%	
%	\subsection{Transform}
%		\subsubsection{Syntax}
%		\subsubsection{Description}
%		\subsubsection{Arguments}
%		\subsubsection{Defaults}
%		\subsubsection{Examples}
%	

	\section{Example Script Files}
	The following file (titled ``Example Script 1'')reads two input sequence files (net-I.fas and net-II.fas), skips all the lines that begin with double dash (\texttt{--}),
	reads the graph file net-I-II.dot, sets the outgroup to the taxon named ``zero,'' specifies the graph type for the analysis is a soft-wired network, and
	reports a series of files with various information about the data and graphs.
	\begin{verbatim}
		-- Example Script 1
		read("net-I.fas")
		--read("net-Ia.fas")
		--read("net-IIa.fas")
		read("net-II.fas")
		--read("net-I.dot")
		--read("net-I.tre")
		--read("net-II.tre")
		--read("net-II.dot")
		read("net-I-II.dot")
		set(outgroup:"zero")
		set(graphtype:softwired)
		report("net-test.tre", graphs, newick, overwrite)
		report("net-test.dot", graphs, dot, overwrite)
		report("net-test-data.csv", data, overwrite)
		report("net-test-diag.csv", diagnosis, overwrite)
		report("net-display.dot", displaytrees, dot, overwrite)
		report("net-display.tre", displaytrees, newick, overwrite)
		
	\end{verbatim}
	
	\section{Execution in Parallel}
	By default the program will execute using a single process core.  By specifying the options `+RTS -NX -RTS' where `X' is the number of processors offered to the program. These are specified after the program as in (for 4 parallel threads):\\
	\\
	PhyGraph +RTS -N4 -RTS other options...  \\
	
	\section*{Acknowledgments}
	The author would like to thank DARPA SIMPLEX N66001-15-C-4039, the  Robert J. Kleberg Jr. and Helen C. Kleberg foundation grant ``Mechanistic Analyses of Pancreatic Cancer Evolution'', and the American Museum of Natural History for financial support.  
	
	\newpage
	%\bibliography{big-refs-3.bib}
	\bibliography{/users/ward/Dropbox/Work_stuff/manus/big-refs-3.bib}
	%\bibliography{/home/ward/Dropbox/Work_stuff/manus/big-refs-3.bib}
\end{document}