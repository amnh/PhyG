\documentclass[11pt]{book}
\usepackage{longtable}
\usepackage{color}
\usepackage{tabu}
\usepackage{setspace}
\usepackage{pdflscape} 
\usepackage{graphicx}
\usepackage {float}
%\usepackage{subfigure}
\usepackage{caption}
\usepackage{subcaption}
\usepackage{natbib}
\usepackage{fullpage}
\bibliographystyle{plain}
%\bibliographystyle{cbe}
\usepackage{algorithmic}
\usepackage[vlined,ruled]{algorithm2e}
\usepackage{amsmath}
\usepackage{amsfonts}
\usepackage{amssymb}
\usepackage[T1]{fontenc}
\usepackage{url}
\usepackage[dvipsnames]{xcolor}
\usepackage{color, soul}
\usepackage[colorlinks=true, linkcolor=blue, citecolor=DarkOrchid, urlcolor=TealBlue ]{hyperref}
%\usepackage[nottoc,numbib]{tocbibind}
\usepackage{tocloft}
\usepackage[many]{tcolorbox}
\usepackage{marginnote}
\usepackage{lipsum}
%\usepackage[most]{tcolorbox}

\setlength\itemindent{1cm}

\newcommand{\phyg}{\texttt{PhyG} }
\newcommand{\atsymbol}{@}
\newenvironment{phygdescription}{\subsubsection{Description}}{}
\newenvironment{example}{\subsubsection{Examples} \begin{itemize}}{\end{itemize}}
\newenvironment{argument}{\subsection{Arguments}\begin{itemize}}{\end{itemize}}
%\item [ ]

\begin{document}
	%\firstpage{1}
	
	\title{PhylogeneticGraph\\User Manual\\Version 0.1}
	
	\maketitle
	
	\newpage

	 \begin{center}
		\includegraphics[width=\textwidth]{AMNHLogo.pdf}
	\end{center}

	\vspace*{2.50cm}	
	\begin{flushleft}
		\textbf {Program and Documentation} \\ Ward C. Wheeler \\
		\vspace*{0.50cm}
		\textbf {Program} \\ Alex Washburn \\
		\vspace*{0.50cm}
		\textbf{Documentation} \\ Louise M. Crowley
	\end{flushleft}
	

	\vspace*{2.50cm}
	
	\begin{flushleft}
		\small
		{\it Louise M. Crowley, Alex Washburn, Ward C. Wheeler} \\
		
		Division of Invertebrate Zoology, American Museum of Natural History, New York, NY, U.S.A.\\
		\smallskip
		The American Museum of Natural History\\
		\copyright 2023 by The American Museum of Natural History, \\
		All rights reserved. Published 2023.
		
		%	\vspace*{0.25cm}
		%	\emph{W. C. Wheeler.} 2022. \texttt{PHYG} 1..0. New York, 
		%	American Museum of Natural History. Documentation by W.C. Wheeler and L. M. Crowley. 
		%	
		\vspace*{0.25cm}
		
		Available online at \url{https://github.com/amnh/PhyGraph}
		
		Comments or queries relating to the documentation should be sent to \href{mailto:wheeler@amnh.org}
		{wheeler@amnh.org} or \href{mailto:crowley@amnh.org}{crowley@amnh.org}
	\end{flushleft}
	
	\tableofcontents

\chapter{What is PhyG?}

\section{Introduction}
	PhylogeneticGraph (\texttt{PhyG}) is a multi-platform program designed to produce phylogenetic 
	graphs from input data and graphs via heuristic searching of general phylogenetic graph 
	space. \texttt{PhyG} is the successor of \href{https://github.com/wardwheeler/POY5}{\textbf{POY}}
	\citep{POY2,POY3,POY4,Varonetal2010,POY5, Wheeleretal2015}, containing much of its 
	functionality, including the optimization of \textit{unaligned} sequences, and the ability to implement 
	search strategies such as random addition sequence, swapping, and tree fusing. As in {\textbf{POY}, 
	\phyg can generate outputs in the form of implied alignments and graphical representations of 
	cladograms and graphs. What sets \phyg apart from {\textbf{POY}, and other phylogenetic 
	analysis programs, is the extension to broader classes of input data and phylogenetic graphs. 
	The phylogenetic graph inputs and outputs of \texttt{PhyG} include trees, as well as softwired 
	and hardwired networks.
		
	This is the initial version of documentation for the program.

%%%%%%%%%%%%%%%%%%%%%%%%%%%%%%%%%%%%%%%%
%QUICKSTART
%%%%%%%%%%%%%%%%%%%%%%%%%%%%%%%%%%%%%%%%		
\section{Quick Start}
	
	\subsection{Requirements: software and hardware}
		\phyg is an open-source program that can be compiled for Mac OSX and Linux 
		(see information relating to Windows machines below). Some utility programs 
		(such as TextEdit for Mac, or Nano for Linux) can help in preparing \phyg scripts 
		and formatting data files, while others (such as Adobe Acrobat and TreeView 
		\citep{page1996}) can facilitate viewing the outputted graphs and trees.
		
		\phyg runs on a variety of computers, including desktops, laptops and cluster computers.
		By default, \phyg is a multi-threaded application and will use the available resources of 
		the computer during the analysis (see Execution in Parallel \ref{subsec:parallel}). 
		
	\subsection{Obtaining and Installing \phyg}
		\phyg source code, precompiled binaries, test data, and documentation in pdf format, 
		as well as tutorials, are available from the \phyg \href{https://github.com/amnh/PhyGraph}{GitHub} 
		website.

	\subsubsection{Installing from the binaries}
		Download the \phyg binary from the \href{https://github.com/amnh/PhyGraph}{GitHub} 
		website. Binaries are available for Mac OSX computers with either Intel or M1 processors, 
		and Linux machines (see information relating to Windows machines below).
		The user can go directly to the website and click on the appropriate link 
		for the binary. On most systems this will download to either your Desktop or Downloads folder. \\
		
		Alternatively, open a \textit{Terminal} window (located in your Applications folder) and type 
		the following for either the Mac Intel, Mac M1 or Linux binary:
		
		\begin {quote}
		curl -LJ --output phyg https://github.com/amnh/PhyGraph/blob/main/bin/OSX/phyg-Intel?raw=true
		\end{quote}		
		
		\noindent or 
		
		\begin {quote}
		curl -LJ --output phyg https://github.com/amnh/PhyGraph/blob/main/bin/OSX/phyg-M1?raw=true
		\end{quote}	
		
		\noindent or 
		
		\begin {quote}
		curl -LJ --output phyg https://github.com/amnh/PhyGraph/blob/main/bin/linux/phyg?
		raw=true
		\end{quote}
		
		\noindent The binary should either be moved into your \$PATH or referred to its 
		absolute when executing a script.\\
		
		For those users with Windows machines, a Windows Subsystem for Linux 
		(WSL) can be installed. This system will allow you to run the Linux binary directly 
		on your machine, without having to install a virtual machine or dual-boot setup. 
		The WSL, along with directions for installation, can be found 
		\href{https://learn.microsoft.com/en-us/windows/wsl/}{here}.
	
	\subsubsection{Compiling from the source}
		For the majority of users, downloading the binaries will suffice. Should the user prefer to 
		compile \phyg directly from the source, the source code can be downloaded 
		from the \href{https://github.com/amnh/PhyGraph}{GitHub} website. \phyg is largely 
		written in Haskell. In order to compile \phyg from the source, the user must install Cabal, 
		a command-line program for downloading and building software written in Haskell (GHC).  
		Information on its installation can be found  
		\href{https://www.schoolofhaskell.com/user/simonmichael/how-to-cabal-install}{here}.
		To compile an optimized version of \texttt{PhyG}, open a \textit{Terminal} and run the 
		following:
		
		\begin{quote}
		%cabal install PhyGraph:phyg --project-file=cfg/cabal.project.release
		cabal build PhyGraph:phyg --flags=super-optimization
		\end{quote}
		
%%%%%%%%%%%%%%%%%%%%%%%%%%%%%%%%%%%%%%%%
%OVERVIEW OF THE PROGRAM
%%%%%%%%%%%%%%%%%%%%%%%%%%%%%%%%%%%%%%%%	 		
\section{Overview of program use}
	At present, \phyg is operated solely via command-line in a \textit{Terminal} window
	and cannot be executed interactively. Commands are entered via a script file 
	containing commands that specify input files, output files and formats, graph type 
	and search parameters.
		
	\subsection{Executing Scripts}
		The program is invoked from the command-line as in:
		
		\begin{quote}
		phyg commandFile
		\end{quote}
		
		\smallskip
		
		\noindent For example, typing the following in a \textit{Terminal} window will invoke 
		\phyg to run the script \texttt{mol.pg}, which is located in the Desktop folder 
		\texttt{phygfiles}:
		
		\begin{quote}
  		phyg /Users/Ward/Desktop/phygfiles/mol.pg
		\end{quote}
		
		\bigskip
		
		\noindent 
		This is the equivalent of typing the following from any location on your computer:
		
		\begin{quote}
   		cd ("/Users/Ward/Desktop/phygfiles")
		\end{quote}
			
		\subsection{Creating and running \phyg scripts}
		A script is a simple text file containing a list of commands to be performed. 
		This script can also include references to other script files 
		(Figure \ref{firstscript}).
		
		Scripts can be created using any conventional text editor such as \textit{TextEdit}, 
		\textit{TextWrangler}, \textit{BBEdit}, or \textit{Nano}. Do not use a word processing 
		application like \textit{Microsoft Word} or \textit{Apple Pages}, as these programs 
		can introduce hidden characters in the file, which will be interpreted by \phyg and 
		can cause unpredictable parsing of the data. Comments that describe the contents 
		of the file and provide other useful annotations can be included. Comment lines are 
		prepended with `-{}-' and can span multiple lines, provided each line begins with `-{}-'. 

%Comemnt

		\begin{figure}[H]
		\centering
		\includegraphics[width=\textwidth]{First_run.jpg}
		\caption{\phyg scripts. The headers in the scripts are comments, leading with `-{}-', 
		which is ignored by \phyg. This first script ``\textbf{First\_script.pg}'' includes a reference 
		to the second script ``\textbf{Chel\_files.txt}'', which includes a group of data files to be 
		read by the program.}
		\label{firstscript}
		\end{figure}
	
	\subsection{Execution in Parallel}
	\label{subsec:parallel}
		\phyg is a multi-threading application and will, by default,  use
		all available cores. Should the user wish to limit or 
		specify the number of processors used by \phyg this can be achieved by including the 
		options `\textbf{+RTS -NX -RTS}', where `\textbf{X}' is the number of processors offered 
		to the program, when executing the script. Should the user wish to use a single processor, 
		this can be specified by typing:

		\begin{quote}
		phyg fileName +RTS -N1 
		\end{quote}
		
		\medskip
		\noindent This will execute the program sequentially.		
		%possible to include and other options

		
%%%%%%%%%%%%%%%%%%%%%%%%%%%%%%%%%%%%%%%%
%FORMATS
%%%%%%%%%%%%%%%%%%%%%%%%%%%%%%%%%%%%%%%%
\section{Input Data Formats} 
	\phyg can analyze a number of different data types, including qualitative, nucleotide
	and other sequences  
	(aligned and unaligned), in TNT, fasta, and fastc formats. Any character names in 
	input files are (for now) ignored and internal names are created by appending 
	the character number in its file to the filename as in ``\textbf{fileName:0}''. 
	Qualitative data, and prealigned data include their indices within input files and 
	unaligned data are treated as single characters.
	
	\phyg allows the user to comment out portions of a taxon name 
	in the imported data file. This is achieved by inserting a dollar sign (`\$') before the region 
	of text that the user wishes to comment out. As an example, placing a `\$' before the 
	GenBank information in the taxon name \textbf{Lingula\_anatina\$\_AB178773\_16S} 
	will comment out this information and the taxon name will be read as 
	\textbf{Lingula\_anatina} by the program. This can be useful for housekeeping purposes, 
	when it is desirable to maintain long verbose taxon names (such as catalog or NCBI 
	accession numbers) associated with the original data files but avoid reporting these 
	names on the graphs. Moreover, it allows the user to provide a single name for a terminal
	in cases where the corresponding data are stored in different files under different terminal
	names.
		
	\subsection{fasta}
		Single character sequence input \citep{PearsonandLipman1988} (see Figure 
		\ref{fasta-c}).
		
	\subsection{fastc}
		Fastc is a file format for multi-character sequence input \citep{WheelerandWashburn2019}.
		This format is derived from the fasta format and the custom alphabet format of 
		 \href{https://github.com/wardwheeler/POY5}{\textbf{POY}} \citep{POY4,POY5}. 
		 Multi-character sequence elements are useful for the analysis of data types such 
		 as developmental, gene synteny, and comparative linguistic data (see Figure 
		 \ref{fasta-c}).  In this format, individual sequence elements are separated by a space.
		 
		\begin{figure}[H]
		\centering
		\includegraphics[width=0.48\textwidth]{fasta.png}
		\includegraphics[width=0.48\textwidth]{fastc.png}
		\caption{Fasta and Fastc file formats. The file ``\textbf{chel\_cox1aln1.fasta}'' 
		represents a fasta file with a greater than sign (`>') preceding taxon names and nucleotide  
		sequence data following on a new line. The file ``\textbf{woman.fastc}'' is a fastc 
		file, with a greater than sign (`>') preceding taxon names and the linguistic data following 
		on a new line.}
		\label{fasta-c}
		\end{figure}
		
	\subsection{\texttt{TNT}}
	\label{subsec:TNT}
		The TNT \citep{Goloboffetal2008} format is accepted here for specification of 
		qualitative, measurement, and prealigned molecular sequence data. \phyg can 
		parse all of the basic character settings including activation/deactivation of 
		characters, making characters additive/non-additive, applying weight to characters, 
		step-matrix costs, interleaved data files and continuous characters (see the 
		argument \texttt{tnt} in \ref{subsec:read}). Continuous characters can only be 
		in the format of integers or floats. Moreover, they must be declared as ``additive'', 
		otherwise, they will be treated as non-additive character states (see 
		\href{http://phylo.wikidot.com/tnt-htm}{\textbf{TNT}}).
		The characters `-' and `?' can be used to indicate that characters or character 
		states are missing or inapplicable. Only one set of ccode commands is allowed 
		per line (e.g. to make everything additive: ccode +.;). Default values of step-matrix 
		weights are not implemented, all step-matrix values must be individually specified. 
		Ambiguities or ranges are denoted in square brackets (`\texttt{[ ]}') with a period, 
		as in [X.Y]. Continuous characters are denoted in square brackets with a dash or 
		hyphen, as in [X-Y].

\section{Input Graph Formats}
	Graphs can be input in a number of formats, including newick, enhanced newick and 
	dot. Newick tree file format is the parenthetical representation of trees as interpreted 
	by Olsen (linked \href{https://evolution.genetics.washington.edu/phylip/newick_doc.html}
	{here}). Enhanced Newick \cite{Cardonaetal2008} is an extension of Newick file 
	format, with the addition of tags (`\#') that indicate hybrid zones or reticulation events
	in the phylogenetic network. \href{https://graphviz.org/}{Dot} is a graph description 
	language and well suited to represent graphs and networks. 	
	%and Forest Enhanced Newick (as defined by Wheeler, 2022 \cite{Wheeler2022}). 
	%Forest Enhanced Newick (FEN) is a format based on Enhanced Newick (ENewick) for 
	%forests of components, each of which is represented by an ENewick string. The ENewick 
	%components are surrounded by `$<$' and '$>$'. As in $<$(A, (B,C)); (D,(E,F));$>$. 
	%Groups may be shared among ENewick components.
	
	\begin{figure}%[H]
	\centering
	\includegraphics[width=\textwidth]{enewick.png}
	\caption{The file \textbf{``flu\_net1.tre''} in Enhanced Newick (ENewick) graph format. The 
	values associated with the taxon names and HTUs are branch lengths. The cost of the 
	graph(s) can be found in square brackets at the end of each graph. A semi-colon follows
	the cost of each graph as a comment.}
	\label{enewick}
	\end{figure}
	
\section{Output Graph Formats}
\label{sec:outputgraphs}
	Graph outputs can be in either newick, enewick, dot, and depending on the 
	operating system, eps or pdf formats. Newick tree files can be viewed in other 
	programs like \href{http://tree.bio.ed.ac.uk/software/figtree/}{FigTree} or 	
	\href{http:/https://uni-tuebingen.de/fakultaeten/mathematisch-naturwissenschaftliche-fakultaet/fachbereiche/informatik/lehrstuehle/algorithms-in-bioinformatics/software/}{Dendroscope}. 
	Enewick files can only be viewed in a text editor. \phyg can output a dot file, 
	along with an eps (on OSX) or pdf (on linux) file that can be viewed in a vector 
	graphics program. The dot file can be viewed (and modified) in \textit{Graphviz}. 
	Note: in order to output pdf files the application \textit{dot} must be installed from 
	the \href{https://graphviz.org/download/}{Graphviz} website. Graphviz an open-source
	graph visualization software. \\	
	
	\noindent PDFs can also be generated directly from dot files. In a \textit{Terminal}, 
	type the following: 

	\begin{quote}
	dot -Tpdf myDotFile.dot $>$ myDotFile.pdf
	\end{quote}
		
	\noindent Multiple `dot' graphs can be output in a single file. To create pdf and 
	other formats the commandline would be (these files are named and numbered 
	automatically):
	
	\begin{quote}
	dot -Tpdf -O myDotFile.dot
	\end{quote}
		
	\noindent In OSX the `pdf' option does not currently seem to work. In this case, 
	type the following:

	\begin{quote}
	dot -Tps2 myDotFile.dot > myDotFile.ps
	\end{quote}
	
	\noindent `-Tps2' will generate a postscript file that \textit{Preview} can read and 
	convert to pdf.

%%%%%%%%%%%%%%%%%%%%%%%%%%%%%%%%%%%%%%%%
%COMMANDS
%%%%%%%%%%%%%%%%%%%%%%%%%%%%%%%%%%%%%%%%
	
\chapter{PhyG Commands}

\section{\phyg Command Structure}
		 
	\subsection{Brief description}
		\phyg interprets and executes scripts coming from an input file. A script is a list of 
		commands, separated by any number of whitespace characters (spaces, tabs, or 
		newlines). Each command consists of a name, followed by an argument or list of 
		arguments separated by commas (`\texttt{,}') and enclosed in parentheses. 
		Commands and arguments are case insensitive with the exception of filename 
		specifications, which must match the filename \textit{exactly}, including suffixes 
		and are always in double quotes (``\textbf{fileName}''). Most commands, with the 
		exception of \texttt{reblock}, \texttt{rename} and \texttt{transform}, have default values 
		and are provided at the end of each command section below. If no arguments are 
		specified, the command is executed under its default settings.\\
		
		\noindent \phyg recognizes three types of command arguments: \textit{primitive values}, 
		\textit{labeled values}, and \textit{lists of arguments}.\\
		
		\noindent \textbf{Primitive values} can either be an integer (\texttt{INT}), 
		a real number (\texttt{FLOAT}), a string (\texttt{STRING}) or a boolean (\texttt{BOOL)} 
		i.e. True|False. \\
		
		\noindent \textbf{Labeled arguments} consist of an argument and an identifier 
		(the label), separated by the colon (`\texttt{:}') character. Examples of 
		identifiers include \texttt{majority}, \texttt{nopenalty}, \texttt{hardwired}, and 
		\texttt{parsimony}. \\
		
		\noindent \textbf{List of arguments} is several arguments enclosed in parenthesis 
		and separated by commas (`\texttt{,}'). Most arguments are optional, with only a 
		few requiring specification, e.g. the build method of distance must be specified. 
		%Some argument labels are obligatory, most are optional \hl{e.g. ?}. 
		In cases where an argument must be specified, \phyg will report a warning message 
		in the output of the terminal window. When no arguments are specified, \phyg will 
		use the default values.\\
		
		\noindent The following examples illustrate the structure of valid \phyg commands. 
					
		\begin{quote}
		build()
		\end{quote}
		
		\noindent In this simple example, the command \texttt{build} is followed by an open 
		and closed parenthesis. As no arguments are specified, \phyg will use the defaults, 
		so this is equivalent to \texttt{build(character, replicates:10)}.	
			
		\begin{quote}
		read(nucleotide:"chel-prealigned.fas", tcm:"sg1t4.mat")
		\end{quote}		

		\noindent In this second example, the command \texttt{read} reads data in the file
		(\texttt{STRING}) ``\textbf{chel-prealigned.fas}'', parsing it as nucleotide sequence data. 
		It also read the data in the file ``\textbf{sg1t4.mat}'', parsing it as a transformation cost matrix, 
		with indel and substitution costs set to $1$.
		
		\begin{quote}
		swap(drift:3, acceptequal:2.0)
		\end{quote}
				
		\noindent In the third example, the command \texttt{swap} is followed by the list 
		of arguments that includes \texttt{drift} and \texttt{acceptequal}, enclosed in 
		parentheses and separated by a comma. Both of these are labeled-value arguments, 
		ascribed an \texttt{INTEGER} and a \texttt{FLOAT}, respectively.
		
		\begin{quote}
		set(graphfactor:nopenalty) 
		\end{quote}
		
		\noindent In this fourth and final example, the command \texttt{set} is followed by the 
		labeled-value argument \texttt{graphfactor} with the label \texttt{nopenalty}.
				
	\subsection{Command order and processing}
		The commands \texttt{read}, \texttt{rename}, \texttt{reblock}, and \texttt{set} are 
		executed at the beginning of program execution, irrespective of where they appear 
		in the command script. All other commands are executed in the order they are specified. 
		
	\subsection{Notation}
		Commands are listed alphabetically in the next section. Commands are shown in 
		\texttt{terminal} typeface. Optional items are enclosed in square brackets (`\texttt{[ ]}').
		Primitive values are shown in \texttt{UPPERCASE}. 
	

%%%%%%%%%%%%%%%%%%%%%%%%%%%%%%%%%%%%%%%%
%COMMAND REFERENCE
%%%%%%%%%%%%%%%%%%%%%%%%%%%%%%%%%%%%%%%%

\section{Command Reference}
	%---------------------------------------------
%build
%---------------------------------------------
\subsection{Build}
	\subsubsection{Syntax}
		\texttt{build(arg0, arg1:option, ...)}\\
		%\texttt{build([argument list])} \hl{is this a simpler way to write it?}
	
	\begin{phygdescription}
		{Builds initial graphs. The arguments of \texttt{build} specify the number of graphs 
		to be generated, and whether the build is based on \textit{distance} or \textit{character} 
		methods. Most options, with the exception of \texttt{rdWag}, an $O(n^3)$ distance based 
		option, are $O(n^2)$. Distance methods are considerably faster (lower constant 
		factor), but approximate in terms of character-based methods. Refinement, in the form 
		of branch swapping (\texttt{none}, \texttt{otu}, \texttt{spr}, and \texttt{tbr}) can be 
		specified within the command for distance builds. Refinement for character-based 
		Wagner builds occurs after the \texttt{build} process through \texttt{swap} and other
		refinement commands. Given the large time burden, distance refinement is usually 
		not time effective \citep{Wheeler2021}. \phyg does not replace trees previously 
		stored in memory.}
	\end{phygdescription}
		
	\subsubsection{Arguments}

	\begin{description}

		\item[block] Performs independent builds for each ``block'' of data. If this option 
		is not specified, the builds are performed combining all the data. Builds are performed 
		according 	to the other options (e.g. \texttt{character, distance}). The resulting tree 
		or \texttt{graph} is reconciled using the \texttt{eun} or \texttt{cun} commands. The 
		reconciled graph is resolved into display trees via the \texttt{displayTrees}, \texttt{first}, 
		and \texttt{atRandom} options. This option is especially useful for soft-wired network search. 
		Associated arguments of \texttt{block} include:
				
		\begin{description}
			\item[atRandom] When the option \texttt{block} is specified, this variable returns $n$ 
			display trees as specified by \texttt{displayTrees[:n]}, where trees are produced by 
			resolving network nodes uniformly at random. Compare with the \texttt{first} option, 
			which takes the ``first'' $n$ display trees resolved in arbitrary, but consistent, order.

			\item[block] Performs independent builds for each ``block'' of data.  These builds are combined
			into a single graph via \texttt{EUN} or \texttt{CUN} and returned as a graph (\texttt{graph}) and/or
			resolved display trees (\texttt{displaytrees})
			
			\item[cun] Reconciles \textbf {block} trees into a Cluster-Union-Network \citep{Baroni2005} 
			before resolution into display trees via the \texttt{displayTrees} or \texttt{atRandom} 
			options.
	
			\item[displayTrees[:n]] When the option \texttt{block} is specified, this variable 
			returns $n$ display trees specified by this optional argument. If the number of 
			display trees is not specified, up to $2^{63}$ may be returned.

			\item[eun] Reconciles block trees into a Edge-Union-Network \citep{MiyagiandWheeler2019, 
			Wheeler2022} before resolution into display trees via the \texttt{displayTrees} or 
			\texttt{atRandom} 
			options.

			\item[first] When the option \texttt{block} is specified, this variable specifies to 
			choose the first returns $n$ $n$ display trees resolved in arbitrary, but consistent, 
			order. Compare with 	\texttt{atRandom}.
			
			\item[graph] When the option \texttt{block} is specified, this variable returns the 
			reconciled graph as specified by \texttt{eun} or \texttt{cun}. The graph may be 
			altered to ensure that it is a ``phylogenetic'' graph sensu \cite{Moretetal2005}.
		\end{description}			
		
		\item [character] Performs random addition sequence Wagner \citep{Farris1970} builds 
		($O(n^2)$) for tree construction. If the graphtype is specified as softwired or hardwired 
		the resulting trees are rediagnosed as soft-wired graphs. This is 
		the default method for tree construction.
		
		\item [distance] Causes a pairwise distance matrix to be calculated ($O(n^2)$) and used 
		as a basis for distance tree construction. Specifies how the builds are refined (\texttt{none}, 
		\texttt{otu}, \texttt{spr}, \texttt{tbr}), as well as how the tree is constructed (\texttt{dWag}, 
		\texttt{nj}, \texttt{rdWag}, \texttt{wpgma}). Associated arguments of \texttt{distance} include:
				
		\begin{description}
			\item[best:n] Applies only to \texttt{rdWag}. Specifies the number of trees retained 
			after 	\texttt{rdWag} builds, selecting the best trees in terms of distance cost. The 
			options can be used to reduce the number of trees retained.  
			This number, $n$, of distance trees are rediagnosed as character trees
			and returned (limited by \texttt{return:m} below) for further analysis.
			
			\item[dWag] Performs distance Wagner build as in \citep{Farris1972} choosing the 
			`best' taxon to 
			add at each step, yielding a single tree. This method has a time complexity of $O(n^3)$.

			\item[nj] Performs Neighbor-Joining distance build \citep{Saitou1987}, yielding a single 
			tree. This method has a time complexity of $O(n^3)$.

			\item[none] No refinement (\texttt{otu}, \texttt{spr}, \texttt{tbr})) is performed after 
			distance builds. \texttt{none} is the default refinement method.
						
			\item[otu] Specifies that \texttt{otu} refinement \citep{Wheeler2021} is performed 
			after distance 
			builds.
			
			\item[rdWag] Performs randoms addition sequence distance Wagner builds, 
			yielding multiple trees determined by the argument \texttt{replicates:n}. This 
			method has a time complexity of $O(m \times n^2)$.
			
			\item[return:n] Limits the number of Wagner trees returned for further analysis.  By default
			all trees that are built (or limited by \texttt{best:m} in distance analysis) are returned.
			Limiting the number of returned trees (as opposed to simply generating that number) can result in
			a larger memory footprint.

			\item[spr] Specifies that \texttt{spr} refinement \citep{Dayhoff1969} is performed 
			after distance builds.

			\item[tbr] Specifies that \texttt{tbr} refinement \citep{Farris1988, swofford1990a} 
			is performed after distance builds.
		
			\item[wpgma] Performs Weighted Pair Group Method with Arithmetic Mean 
			distance build \citep{SokalandMichener1958}, yielding a single tree. This method 
			has a time complexity of $O(n^2)$.
		\end{description}

		\item [replicates:n] Applies to \texttt{rdWag} and \texttt{character}. Specifies the number of 
		random addition sequences performed.
	
	\end{description}		

	\subsubsection{Defaults}
		\texttt{build(character, replicates:10)}
		
	\begin{example}
	
		\item{\texttt{build(replicates:100)} \\
		Builds 100 random addition sequence character Wagner builds.}
		
		\item{\texttt{build(character, block, graph, cun, displaytrees:5, atrandom)}\\
		Builds 10 (the default) random addition sequence character Wagner builds, for each 
		block of data. The graph is reconciled into a Cluster-Union-Network, before resolution 
		into the 5 display trees. The trees are produced by resolving the network nodes 
		uniformly atRandom.}
		
		\item{\texttt{build(distance, rdWag, nj, wpgma)} \\ 
		Builds a single `best' addition sequence distance Wagner build, a Neighbor-Joining tree, 
		and a WPGMA tree. As the option \texttt{block} is not specified, the distance trees are 
		built using all the data.}
		
		\item{\texttt{build(distance, dWag, replicates:1000, best:10)}\\
		Builds 1000 distance Wagner builds and returns 10 of the lowest cost distance trees.}
	
		\item{\texttt{build(distance, rdwag, block, eun, displaytrees:3)}\\
		Builds 10 random addition sequence Wagner builds for each `block' of data. The graph 
		is reconciled into a Edge-Union-Network, before resolution into the 3 display trees. 
		The trees are produced by resolving the network nodes.}
		
		\item{\texttt{build(distance, block, rdWag, replicates:100, wpgma, best:10, otu)}\\
		Builds 100 random addition sequence distance Wagner builds, a WPGMA tree, 
		performs OTU swapping on the WPGMA and 10 of the lowest cost random addition 
		sequence Wagner trees. This distance search is performed on the `blocks' of data, 
		as opposed to all of the data.}

	\end{example}

%---------------------------------------------
%fuse
%---------------------------------------------
\subsection{Fuse}
	\subsubsection{Syntax}
		\texttt{fuse(option, option, ...)}
		
	\begin{phygdescription}
		{Performs Tree Fusing \citep{goloboff1999, moilanen1999, moilanen2001}. \texttt{fuse} 
		operates on a collection of graphs performing reciprocal graph recombination between 
		pairs of graphs. Non-identical subgraphs with identical leaf sets are exchanged between 
		graphs and the 	results evaluated. This process of exchange and evaluation continues 
		until no new graphs are found.
		This can be used in concert with other options to perform a Genetic Algorithm refinement
		\citep{Holland1975}. The behavior of \texttt{fuse} can be modified by the use of options 
		specifying SPR and TBR-like rearrangement of the combination process.}
	\end{phygdescription}
	
	\subsubsection{Arguments}
	\begin{description}
		\item [all]  During branch swapping-type operations, all rearrangement are tried before choosing a new graph. 
		
		\item [alternate[:n]] Causes the exchanged subgraphs to be tried at multiple positions including rerooting of 
		exchanged groups--alternating between SPR and TBR-type edits (up to optional 
		$n$ edges aware from their initial positions).
		
		\item [atRandom] Chooses graphs to fuse uniformly at random when \texttt{pairs} is specified. 
		
		\item [best] Specifies the method for tree selection, which in this case returns the best graphs 
		found during fuse operations.		
		
		\item [keep:n] Limits the number of returned graphs to $n$. 
		
		\item [nni] Causes the exchanged subgraphs to be tested at their initial positions as well as the 
		two adjacent edges.
		
		\item [notReciprocal] Turns off \texttt{reciprocal}.
		
		\item [once] Performs a single round of fusing on input graphs and returns the resulting graphs. 
		Alternatively (and by default) fusing continues recursively until no new graphs are found.
		
		\item [pairs[:n]] Limits the number of graphs to be fused to $n$ pairs (as oppose to $\binom{m}{2}$ for $m$ graphs).
		
		\item [reciprocal] By default, fuse takes a subgraph of one graph in a pair and replaces the corresponding subgraph 
		in the other.  Reciprocal causes exchange and evaluation or graphs in both directions--roughly doubling time and 
		memory footprint.
		
		\item [spr[:n]] Causes the exchanged subgraphs to be tried at multiple positions (up to optional 
		$n$ edges aware from their initial positions).
		
		\item [steepest] During branch swapping-type operations, if a better graph is found, swapping shifts greedily to
		that graph. 
		
		\item [tbr[:n]] Causes the exchanged subgraphs to be tried at multiple positions (up to optional 
		$n$ edges aware from their initial positions) and with TBR-style rerooting of the exchanged 
		components.
		
		\item [unique] Specifies the method for tree selection, which in this case returns all unique 
		graphs found during fuse operations.	
	\end{description}	
	
	\subsubsection{Defaults}
		\texttt{fuse()} Keeps all the best graphs found, continues fusing until no new graphs are found. 
		No branch swapping style rearrangements are performed.
	
	\begin{example}
	
		\item{\texttt{fuse(best, once)}\\Fuses input graphs and returns best graphs after a single round of 
		fusing.}
		
		\item{\texttt{fuse(tbr, keep:10)} \\Fuses input graphs and preforms TBR-style replacement and 
		rerooting of pruned components returning up to 10 best cost graphs.}
		
	\end{example}

%---------------------------------------------
%read
%---------------------------------------------	
\subsection{Read}
	\label{subsec:Read}
	\subsubsection{Syntax}
		\texttt{read(option:"filename", option:"filename", ...)}
			
	\begin{phygdescription}
		{Imports file-based information, including data files, tree and graph files. \texttt{read} 
		commands must contain an input file. Supported data formats include FASTA, FASTC
		and TNT files, and graph formats include Dot, Enewick, Fenewick, and Newick. 
		Filenames must be included in quotes and, if multiple filenames are specified, 
		separated by commas. Filenames must include the appropriate suffix (e.g. .fas, 
		.ss, .mat). The exclusion of these suffixes will result in an error. The filename must
		match exactly, including capitalization. \phyg will attempt to recognize the type of input
		and parse appropriately. Otherwise, the type of file can be indicated, using one of the 
		options below. The options prepend 	the filename with a colon (`:') and will modify the 
		processing of the input file. Prepending the file type prevents any ambiguity when the 
		file is parsedThis command can also use wildcard expressions (`*', `?'), 
		which can be useful when reading in multiple files of the same type.}
	\end{phygdescription}

	\subsubsection{Arguments}
	
	\begin{description}
		\item [aminoacid:] Specifies that the file contents are parsed as IUPAC coded amino 
		acid data in fasta \citep{PearsonandLipman1988} format.

		\item [block:] Specifies that the file contains block %\hl{ref?} 
		information. Each line contains 
		the new block name followed by names of input files to be assigned to that data block. 
		Blocks are initially named as the input file name with ``:0'' appended. In the examples, 
		data from files ``b'' and ``c'' will be assigned to block ``a''. There can be no spaces in 
		file or block names.
			
			\begin{quote}
					\texttt{"a" "b:0" "c:0"}
			\end{quote}
	
		\item [dot:] Specifies that the file contains a graph in `dot' format for use with graph 
		rendering software such as \href{https://en.wikipedia.org/wiki/Graphviz}{GraphViz}.
			
		\item [enewick:] Specifies that the file contains Enhanced Newick format graph(s) as
		specified here \citep{Cardonaetal2008}. 
			
		\item [exclude:] Specifies that the file contains the names of terminal taxa to be 
		excluded from an analysis. Taxa appear in the form of a list, with a single taxon per 
		line. Thus, taxa not included in the list and present in input files, will be included in 
		analysis. Compare with \texttt{include}.
			
		\item [fasta:] Ensures that file contents are parsed in fasta \citep{PearsonandLipman1988}
		format. This is used for single character sequences such as binary streams, IUPAC 
		nucleotide and amino acid sequence data.
			
		\item [fastc:] Ensures that file contents are parsed in fastc \citep{WheelerandWashburn2019}
		format. This is used for multi-character sequences such as gene synteny, developmental, 
		or linguistic data.
			
		\item [fenewick:] Specifies that the file contains Forest Enhanced Newick format graph(s)  
		specified \href{https://www.github.com/wardwheeler/euncon}{here} \citep{Wheeler2022}.
			
		\item [include:] Specifies the names of terminal taxa to be included in the analysis. 
		Taxa appear in the form of a list, with a single taxon per line. It is possible to specify 
		terminals that have no data. This may be done in order to diagnose a large tree on 
		partial data. If there are no data for a leaf taxon, a warning will be printed to \texttt{stderr}. 
		Taxa not include in this list, but present in the inputted data files, will be excluded from 
		the analysis. Compare with \texttt{exclude}.
			
		\item [newick:] Specifies that the file contains Newick format graph(s) as specified 
		\href{https://evolution.genetics.washington.edu/phylip/newick_doc.html}{here}.
			
		\item [nucleotide:] Ensures that file contents are parsed as IUPAC coded nucleotide data 
		in fasta \citep{PearsonandLipman1988} format.
			
		\item [preaminoacid:] Specifies that the file contents are parsed as IUPAC coded amino 
		acid data in  prealigned fasta \citep{PearsonandLipman1988} format, leaving 
		gap characters (``-'') in the sequences and alignment correspondences are not re-examined.
		
		\item [prefasta:] Specifies that the sequences are prealigned in a fasta format, leaving 
		gap characters (``-'') in the sequences and alignment correspondences are not re-examined. 
		This option exists to ensure proper parsing (and in case auto-format detection is incorrect).
		Prealigned fasta files \textbf{must} be of the same length.
			
		\item [prefastc:] Specifies that the sequences are prealigned in a fastc format, leaving gap 
		characters (``-'') in the sequences and alignment correspondences are not re-examined. 
		This option exists to ensure proper parsing (and in case auto-format detection is incorrect).
		Prealigned fastc files \textbf{must} be of the same length.			
		
		\item [prenucleotide:] Ensures that file contents are parsed as IUPAC coded nucleotide data 
		in fasta \citep{PearsonandLipman1988} format, leaving 
		gap characters (``-'') in the sequences and alignment correspondences are not re-examined.
		
		\item [rename:] Replaces the name(s) of specified terminals in the file. This command allows 
		for substituting taxon names and can help merge multiple datasets without modifying the 
		original data file. The file contains a series of lines, each of which contains at least two strings. 
		The first string (input taxon name) will replace the second and all subsequent strings (taxon 
		names) on that line. In the example given in Figure \ref{renamefile} the taxon Hydrus\_granulatus 
		will be renamed as Acrochordus\_granulatus, the taxa Gloydius\_boehmei and Gloydius\_mogoi
		will be renamed as Gloydius\_halys and the taxa Crotalus\_mutus, Scytale\_catenatus and 
		Coluber\_crotalinus will be renamed as Lachesis\_muta.
		
		\begin{figure}
		\centering
		\includegraphics[width=0.8\textwidth]{Rename_file.jpg}
		\caption{Renaming text file containing the lists of terminal taxa to be renamed.}
		\label{renamefile}
		\end{figure}
		
		The \texttt{rename} function can also be specified as a command, see \texttt{rename}
		(Section \ref{subsec:Rename}).
					 
		\item [tcm:] This refers to a file containing a custom-alphabet matrix that specifies varying 
		costs among alphabet elements in a sequence. The elements in the alphabet can be letters, 
		digits, or both. \\
		The \texttt{tcm} contains two parts: the first line of the file contains the alphabet elements 
		separated by a space and the transformation cost matrix, which follows below. The dash 
		character representing an insertion/deletion or indel character is not specified on the first 
		line of the file, but added to the alphabet automatically. The second part is the \texttt{tcm}, 
		which is a square matrix with $n + 1$ elements ($n$ is the size of the alphabet). 
		The positions from left to right and top to bottom in this matrix correspond to the sequence 
		of the elements as they are listed in the alphabet. An extra rightmost column and lowermost
		row correspond to indel (gap) costs to and from alphabet elements. At present, this matrix 
		must be symmetrical, but not necessarily metric. Non-metric tcm's can yield unexpected 
		results. Transformation costs must be integers. If real values are desired, a character can 
		be weighted with a floating point value factor. \\
		For a sequence with four elements alpha, beta, gamma and delta and an indel cost of 4 
		for all insertion deletion transformations, a valid custom alphabet file is provided below:
		\\
		\begin{equation*}
		%\nolabel
		\begin{array}{lllll}
		alpha & beta & gamma & delta &  \\
		0 &   2 &  1 &   2 &   5 \\
		2 &   0 &  2 &   1 &   5 \\
		1 &   2 &  0 &   2 &   5 \\
		2 &   1 &  2 &   0 &   5 \\
		5 &   5 &  5 &   5 &   0
		\end{array}
		\end{equation*} 
		\\
		In this example, the cost of transformation of \texttt{alpha} into \texttt{beta} is \texttt{2},
		and cost of a deletion or insertion of any of the four elements costs \texttt{5}.

		\item [tnt:] Ensures that file contents are parsed in TNT \citep{Goloboffetal2008} format. 
		Not all TNT data commands are currently supported. To ensure that the file is correctly
		parsed, the file must begin with \texttt{xread}, followed by an optional comment in single 
		quotes ('this is a comment'), followed by the number of characters and taxa. The data 
		follow on a new line. Taxon names are followed by state data. Data may be in multiple 
		blocks (interleaved) or in sequential format. These interleaved blocks may consist of a 
		series of single character states without spaces between them, or multiple (or single) 
		character states (e.g. \texttt{alpha}) with space between the individual codings. Blocks 
		must be of all one type (ie. single character codings without spaces, or multi-character 
		separated by spaces). The data block \textit{must} be followed by a single semicolon 
		(``;'') 	on its own line.\\
			
		The character settings (i.e. \texttt{ccode} commands) follow the data block, beginning 
		on a 	new line. These character settings always terminate with a semi-colon (\texttt{;}). 
		These settings include: activate (\texttt{[}) or deactivate (\texttt{]}); make additive/ordered 
		(\texttt{+}) 	or non-additive/unordered (\texttt{-}); apply step-matrix costs (\texttt{(}) with 
		scopes (e.g. \texttt{cc + 10 12;} and  \texttt{cc (.; costs 0 = 0/1 1 0/2 2 0/3 3 0/4 4 1/2 1 
		1/3 2 1/4 3 2/3 1 2/4 2 3/4 1;}) including abbreviated scopes (\texttt{cc -.;}). %\hl{what about ) make additive or non-additive again?} 
		There may 
		be multiple character setting statements in a single line. Character settings must be 
		followed by \texttt{proc/;} on its own line. \texttt{PhyG} will not process
		any file contents that follow \texttt{proc/;}.
		  
		 Additive/ordered character states must be numbers (integer or floating point). Ranges 
		 for continuous characters are specified with a dash within square brackets (e.g. 
		 \texttt{[1-2.1]}). Character state polymorphism are specified in square brackets without 
		 spaces for single character states (e.g. \texttt{[03]}), and with spaces for multi-character 
		 states. %(e.g. \hl{\texttt{[???}}).
		  
		 Dashes in multi-character states (e.g. \texttt{Blue-ish}) 
		 are treated as part of the character state specification. If the user wishes that dashes 
		 be treated as missing data (`?'), the file must be edited to reflect this by replacing the 
		 dashes that are to be treated as missing data with question 
		 marks (`?').
		  
		  Example file:
		  	\begin{quote}
			  	\texttt{xread\\
				  	`An example TNT file' 8 5\\
				  	A 000\\
				  	B a14\\
				  	C b22\\
				  	D ?33\\
				  	E d[01]4\\}
			  	
			  	\texttt{A Blue-ish -\\
				  	B Green-ish OneFish\\
				  	C Rather-Red TwoFish\\
				  	D Almost-Cyan RedFish\\
				  	E Orange-definitely BlueFish\\}
					
				\texttt{A 5.2 - ?\\
					 B 5.3 0.3 1.1\\
					 C 3.2 0.1 1.1\\
					 D 5.2 1.1 0.1\\
					 E 5.1 1.1 0.1\\
				  	;\\
				  	cc .;\\
				  	cc + 2;\\
				  	proc/;\\}
			  \end{quote}
	\end{description}	
		
	\subsubsection{Defaults}
		\texttt{read("fileName")} reads data from fileName and attempts to recognize the 
		file type and parse accordingly. The assumed file type is printed to \texttt{stderr} for 
		verification.
		
	\begin{example}
		
		\item{\texttt{read(prefasta:"myDnaSequenceFile.fas")}\\ Reads sequence data from 
		``myDnaSequenceFile.fas'' as prealigned data.}
		
		\item{\texttt{read(rename:"myRenameFile")}\\ Reads a list of taxa and names to be 
		assigned.} 
		
	\end{example}
		
%---------------------------------------------
%reblock
%---------------------------------------------
\subsection{Reblock}
	\subsubsection{Syntax}
		\texttt{reblock("newBlockName", "inputFile0", "inputFile1",...)}
	
	\begin{phygdescription}
	{Assigns input data to ``blocks'' that will follow the same display tree when optimized
	as ``soft-wired'' networks. By default, each input data file is assigned its own block with 
	the name of the input file. The \texttt{block} command is used to reassign these data to 
	new, combined blocks. Spaces are not allowed in block names and will produce 
	\texttt{unrecognized block name} errors.} 
	\end{phygdescription}
	
	\subsubsection{Arguments}
		The first argument is the block to be created, the remainder are the input data to 
		be assigned to that block. Blocks are initially named as the input file name with 
		``:0'' appended. Blocks are reported in \texttt{report(data)} command.
	
	\subsubsection{Defaults}
		None.
	
	\begin{example}

		\item{\texttt{reblock("a","b\#0","c\#0")}\\ Assigns input data from file ``b'' and ``c'' 
		to block ``a''. }
	
	\end{example}

%---------------------------------------------
%rename
%---------------------------------------------	
\subsection{Rename}
	\label{subsec:Rename}
	\subsubsection{Syntax}
		\texttt{rename("newName", "oldName1", "oldName2",...)}
		
	\begin{phygdescription}
	{Replaces the name(s) of specified terminals. 
	The command consists of a terminal identifier followed by a comma and then by either 
	a string containing a pair (or pairs) of strings containing the names of items being renamed.
	The first string (name) is assigned to taxa with the strings (names) that follow. This can be 
	useful when combining data from different sources, such as GenBank, or in revising names 
	to reflect taxonomic changes. Irrespective of where this command appears in the script file, 
	\phyg will execute this command prior to importing the data files. Compare with the 
	argument \texttt{rename} of the command \texttt{read} (Section \ref{subsec:Read}).
	In the example below, the taxa ``b'' and ``c'' will be renamed to ``a''.

	\begin{quote}
	rename("a","b","c")
	\end{quote}}
	\end{phygdescription}
	
	\subsubsection{Arguments}
		Taxon names to assign and be assigned.
		
	\subsubsection{Defaults}
		None.
		
	\begin{example}
	
		\item{\texttt{rename("a","b","c")}\\ Renames ``b'' and ``c'' to ``a''. }
				
	\end{example}

%---------------------------------------------
%refine
%---------------------------------------------		
\subsection{Refine}
	\subsubsection{Syntax}
		\texttt{refine(option, option,...)}
		
	\begin{phygdescription}
		{Performs edit operations (addition, deletion, move) on network edges, and therefore
		only applies to soft-wired and hard-wired graphs.}
	\end{phygdescription}

	\subsubsection{Arguments}
		
		\begin{description}
		\item[annealing[:n]] Specifies that $n$ (default 1) rounds of simulated annealing optimization 
		\citep{Metropolisetal1953, Kirkpatricketal1983, Cerny1985} are performed in concert with 
		\texttt{netAdd}, \texttt{netDel}, and \texttt{netMove}. The acceptance of candidate graphs is 
		determined by the probability $e ^ {- (c_c - c_b)/ (c_b * (k_{max} -k)/ k_{max})}$, where $c_c$ 
		is the cost of the candidate graph, $c_b$ is the cost of the current best graph, $k$ is the step 
		number, and $k_{max}$ is the maximum number of steps (set by the \texttt{steps:m}, default 10) 
		option.
		
			\begin{description}
		
			\item[netadd] Adds network edges to existing input graphs at all possible positions until no 
			better cost graph is found.
			
			\item[netdel] Deletes network edges from input graphs one at a time until no better cost 
			graph is found.
			
			\item[netmove] Moves existing network edges in input graphs one at a time to new positions 
			until no better cost graph 
			is found.
			
			\item[steps:n] Specifies that $n$ (default 10) temperature steps are performed during 
			simulated 
			annealing (as specified by the \texttt{annealing}) option.

			\end{description}

			
		\item[drift[:n]] Specifies that $n$ (default 1) rounds of the ``drifting'' form of simulated annealing 
		\citep{goloboff1999} optimization are performed in concert with \texttt{netAdd}, 	\texttt{netDel}, 
		and \texttt{netMove}. The acceptance of candidate graphs is determined by the probability 
		$1/ (wf + c_c - c_b)$, where $c_c$ is the cost of the candidate graph, $c_b$ is the cost of the 
		current best graph, and $wf$ is the \texttt{acceptWorse} (set by the \texttt{acceptWorse:m}, 
		default 1.0) option. Equal cost graphs are accepted with probability set by the \texttt{acceptEqual} 
		option. \texttt{Drift} differs from \texttt{annealing} in that there are no cooling steps to modify 
		acceptance probabilities. The maximum number of graph changes is set by \texttt{maxChanges}
			
					
		\item[geneticAlgorithm or ga] Performs Genetic Algorithm \citep{Holland1975} refinement in 
		concert with the options \texttt{generations}, \texttt{popsize}, \texttt{severity}, and 
		\texttt{recombinations}. 
			
			\begin{description}
		
			\item[generations:[n]] Specifies the number of generations (sequential iterations) for 
			\texttt{geneticAlgorithm}. The default is $n=10$.

			\item[popsize:[n]] Specifies the population size for \texttt{geneticAlgorithm}. The default is 
			$n=20$.
			
			\item[recombinations:[n]] Specifies the number of recombination (fusing) events for 
			\texttt{geneticAlgorithm}. The default is $n=100$.
			
			\item[severity:[n]] Specifies the severity of selection against sub-optimal graph solutions 
			events for \texttt{geneticAlgorithm}. The higher the value, the less severe the penalty. The 
			default is $n=1.0$.
			
			\item[stop:n] Causes the \texttt{geneticAlgorithm} to terminate after $n$ generations without improvement 
			in graph cost.  Default is to only terminate when the number generations has been completed.

			\end{description}
		
		\item[keep:n] Limits the number of returned graphs to $n$. 
		
		\end{description}
	
	%add defaults and examples	
%---------------------------------------------
%report
%---------------------------------------------				
\subsection{Report}
	\subsubsection{Syntax}
		\texttt{report("filename", arg0, arg1,...)}
	
	\begin{phygdescription}
		{Outputs the results of the current analysis to a file or to \texttt{stderr}. To redirect the 
		output to a file, the file name (in quotes), followed by a comma, must be included in 
		the argument list of report. All arguments for \texttt{report} are optional. This command 
		allows the user to output information concerning the characters and terminals, 
		diagnosis, export static homology data, implied alignments, trees, graphs, dot files, 
		as well as other miscellaneous arguments. By default, new information printed to 
		a file if appended to the file. The option \texttt{overwrite} overrides the default and 
		rewrites the file. Many of the report options can be output in csv format,  which can
		subsequently be imported into spreadsheet programs.}
	\end{phygdescription}
	
	\subsubsection{Arguments}
	\begin{description}
		
		\item[crossrefs] Reports a table with terminals represented in rows, and the data files in 
		columns. A plus sign (``+'') indicates that data for a given terminal is present in the 
		corresponding file; a minus sign (``--'') indicates that it is absent. It is highly recommended 
		that the user use this report option to examine the data, having imported them into \phyg. 
		This argument is a useful tool for visual representation of missing data, as well as highlight 
		inconsistencies in the spelling of taxon names in different data files. The reported file is 
		in csv format.
			
		\item[data] Outputs a summary of the input data. More specifically, \phyg will report 
		aspects of the input data. 
		%\hl{"Index","Block","Name","Type","Activity","Weight","Prealigned","Alphabet","TCM"}
	
		\item[diagnosis] Outputs graph diagnosis information such as vertex and states and edge 
		statistics in csv format. 
		
		\item[displaytrees] Outputs graph information for soft-wired networks. The ``display'' trees 
		are output for each data block. 
		
		\item[graph] Outputs a graph in format specified by the other arguments in the command. 
		These are \texttt{dot} for 
		GraphViz graph format, \texttt{dotpdf} for pdf (or ps for OSX), \texttt{newick} for Newick, 
		ENewick, or ForestEnewick depending on the graph type, \texttt{ascii} for an ascii rendering. 
		In order to output pdf files (via \texttt{dotpdf}) ``dot'' must be installed from 
		\url{https://graphviz.org/download/}. PhyG will not error, but output a ``dot'' file that 
		can be processed later.
		
		\item[pairdist] Outputs a taxon pairwise distance matrix in csv format. 
		
		\item[reconcile] Outputs a single ``reconciled'' graph from all graphs in memory. The 
		methods include consensus, supertree, and other supergraph methods as described in 
		\cite{Wheeler2012, Wheeler2022}. When \texttt{reconcile} is specified as a command 
		option a series of other options may be specified to tailor the desired outputs:
			\begin{description}
			\item {Method:eun$\mid$cun$\mid$majority$\mid$strict$\mid$Adams\\Default:eun\\
			This commands specifies the type of output graph. EUN is the Edge-Union-Network 
			\citep{MiyagiandWheeler2019}, CUN the Cluster Union Network \citep{Baroni2005},
			majority (with fraction specified by `threshold') specifies that a values between 0 and 
			100 of either vertices or edges will be retained. If all inputs are trees with the same leaf 
			set this will be the Majority-Rule Consensus \citep{MargushandMcMorris1981}.
			Strict requires all vertices be present to be included in the final graph. If all inputs are 
			trees with the same leaf set this will be the Strict Consensus \citep{Schuhandpolhemus1980}. 
			Adams denotes the Adams II consensus \citep{Adams1972}.}
							
			\item{Compare:Combinable$\mid$identity\\Default:combinable\\Species how group 
			comparisons are to be made. Either by identical match [(A, (B,C))$\neq$(A,B,C)],
			combinable sensu \cite{Nelson1979} [(A, (B,C)) consistent with (A,B,C)]. This option 
			can be used to specify ``semi-strict'' consensus \citep{Bremer1990}.}
							
			\item{Threshold:(0-100)\\Default:0\\Threshold must be an integer between 0 and 100 
			and specifies the frequency of vertex or edge occurrence in input graphs to be included 
			in the output graph. Affects the behavior of `eun' and` majority.'}
			
			\item{Connect:True$\mid$False\\Default:True\\Specifies the output graph be connected 
			(single component), potentially creating a root node and new edges labeled with ``0.0''.}
			
			\item{EdgeLabel:True$\mid$False\\Default:True\\Specifies the output graph have edges 
			labeled with their frequency in input graphs.}
			
			\item{VertexLabel:True$\mid$False\\Default:False\\Specifies the output graph have vertices 
			labeled with their subtree leaf set.}
	
			\item{OutFormat:Dot$\mid$FENewick\\Default:Dot\\Specifies the output graph format 
			as either Graphviz `dot' or FEN.}
			\end{description}	
				
		\item[support] Outputs support graphs. Resampling graphs are independent of the 
		input graphs while Goodman-Bremer are based on current graphs. Multiple formats 
		can be output via additional options including \texttt{dot} for GraphViz graph format, 
		\texttt{dotpdf} for pdf (or ps for OSX), \texttt{newick} for Newick, ENewick, or 
		ForestEnewick depending on the graph type, \texttt{ascii} for an ascii rendering. 
		In order to output pdf files (via \texttt{dotpdf}) ``dot'' must be installed from 
		\url{https://graphviz.org/download/}. \phyg will not error, but output a ``dot'' file that 
		can be processed later.
		
		\item[search] Outputs search statistics in csv format.
		 
	\end{description}			
		
	\subsubsection{Defaults}
		\texttt{report()} prints input data and output graph diagnosis to stderr. Default graph 
		representation is \texttt{dot}.
		
	\begin{example}
		\item{\texttt{report("outFile", newick, overwrite)}\\ Outputs graphs in newick format to 
		``outFile'', overwriting any existing information.}
		
		\item{\texttt{report("outFile", crossrefs)}\\ Outputs presence/absence for taxa in input files. 
		A `+' is output if taxa are present in an input data file, and `-' if. File is in csv format. This 
		can be useful in checking for missing sequence or other data and expected renaming.}
		
		\item{\texttt{report("outFile", dot, reconcile, method:eun, threshold:51)}\\ Outputs reconciled 
		graph using the Edge-Union-Network method with a minimum edge frequency of 51\% in 
		dot format to ``outFile'', appending to any existing information in ``outFile''.}
	\end{example}

%---------------------------------------------
%run
%---------------------------------------------		
\subsection{Run}
	\subsubsection{Syntax}
		\texttt{run("filename")}
		
	\begin{phygdescription}
		{Used to execute a \phyg script file containing commands. The script filename must be 
		included in quotes. 
		Executing scripts using \texttt{run} can be useful to specify common actions such as file inputs 
		and graph construction. }
	\end{phygdescription}
	
	\subsubsection{Arguments}
		The only argument is the filename containing commands to be executed.
		
	\subsubsection{Defaults}
		There are no default settings of \texttt{run}. 
	
	\begin{example}
		\item{\texttt{run("readFiles.pg")}\\ Executes "readFiles.pg", which may contain multiple input 
		files to be \texttt{read}}
		
		\item{\texttt{run("searchCommands.pg")}\\ Executes "searchCommands.pg", which may 
		contain commands defining a common search strategy (e.g. \texttt{build}).}
	\end{example}

%---------------------------------------------
%search
%---------------------------------------------		
\subsection{Search}
	\subsubsection{Syntax}
		\texttt{search(arg0:option, arg1:option, ...)}
	
	\begin{phygdescription}
		{This command implements a default search strategy, performing a timed randomized 
		series of graph optimization methods including building, swapping, recombination (fusing), 
		simulated annealing and drifting, network edge addition/deletion/moving, and Genetic 
		Algorithm. The parameters and order for this search are randomized. The arguments 
		specify the number independent instances of search (\texttt{instances}, default 1), and duration 
		(\texttt{days}, \texttt{hours}, \texttt{minutes}, and \texttt{seconds}; 
		default 30 seconds). Successive rounds of search gather 
		any solutions from previous sequential or parallel rounds as well as any input graphs.
		Since search methods may vary in how long they take, individual iterations may take 
		longer that the specified duration.  By default, search strategies are chosen uniformly at random, 
		if \texttt{Thompson} is specified, Thompson sampling \cite{Thompson1933,WheelerThompson} is used to modify 
		the probabilities of search strategies over the search duration.
		
		When performing a \texttt{search} it is important to set the amount of time, such that the 
		program has a reasonable amount of time to perform a search. Therefore, it is important
		to have some idea as to the length of time it would take to do a single round of searching.}
	\end{phygdescription}
			
	\subsubsection{Arguments}
	\begin{description}
		\item[keep:n] Keeps up to $n$ graphs.
		
		\item[days:n] Adds $n$ 24 hour days to search time.
		
		\item[hours:n] Adds $n$ hours to search time.
		
		\item[maxNetEdges:n] Limits maximum number of network edges to $n$.
		
		\item[minutes:n] Adds $n$ minutes to search time.
		
		\item[seconds:n] Adds $n$ seconds to search time.
		
		\item[instances:n] Specifies $n$ (potentially parallel) search instances.
		
		\item[stop:n] Specifies that the search will be terminated after $n$ iterations without improving the best graph cost.
		
		\item[Thompson] Specifies that randomized choice of search option (e.g. Wagner Build, SPR, Genetic Algorithm) 
		employs Thompson sampling.
		
		\item[mFactor:n] Species the memory factor $n$ for Thompson sampling.
		
		\item[linear] Specifies that the Thompson memory is a linear function of \texttt{mFactor}, $m$.  Updating of search type, 
		$\theta^k$, probability for iteration $n$ is $\frac{m}{m+1} \theta^k_{n-1} + \frac{1}{m+1} \theta^k_n$.  Thompson success
		is a function of whether a search was successful in reducing the graph cost and how long that search took to complete.
		
		\item[exponential] Specifies that the Thompson memory is an exponential function of \texttt{mFactor}, $m$.  Updating of search type, 
		$\theta^k$, probability for iteration $n$ is $ \left(1 - \frac{1}{2^m} \theta^k_{n-1}\right) + \left(\frac{1}{2^m} \theta^k_n \right)$.  Thompson success
		is a function of whether a search was successful in reducing the graph cost and how long that search took to complete.
		
		%\hl{would it not be simpler to do a max\_time:FLOAT:FLOAT:FLOAT, or min\_time
		%FLOAT:FLOAT:FLOAT like we had in poy? How useful are seconds?}
	\end{description}
		
	\subsubsection{Defaults}
		\texttt{search()} Performs 1 instance of 30 seconds keeping up to 10 graph}.
		
	\begin{example}
		\item{\texttt{search(hours:10, instances:2)}\\ Performs 2 search instance (in parallel if the 
		program is executed in parallel) for 10 hours each.}
				
		\item{\texttt{search(hours:10, minutes:30)}\\ Performs a single search instance for 10 
		hours and 30 minutes.}
	\end{example}
	
%---------------------------------------------
%select
%---------------------------------------------		
\subsection{Select}
	\subsubsection{Syntax}
		\texttt{set(arg0:option, arg1:option, ...)}
	
	\begin{phygdescription}
		{Specifies the method and number of graphs to be saved at any point. When multiple 
		graphs are present, the \texttt{select} command will specify which of the graphs to keep 
		for further analysis or reporting.}
	\end{phygdescription}
				
	\subsubsection{Arguments}
		\begin{description}
			\item[all] Keeps all graphs.
		
			\item[best:n] Selects and keeps the graphs with the  best optimality value.
			
			\item[atRandom] Keeps graphs chosen at random.
			
			\item[unique:n] Keeps up to \texttt{n} unique graphs, which may not necessarily be the 
			best graphs.
		\end{description}

	\subsubsection{Defaults}
		\texttt{select()} Keeps all unique graphs of best optimality value.
		
	\begin{example}
		\item{\texttt{select(random:10)}\\ Keeps up to 10 graphs, selected at random.}
						
		\item{\texttt{select(best:10)}\\ Keeps up to 10 graphs of best optimality value.}
	\end{example}

%---------------------------------------------
%set
%---------------------------------------------				
\subsection{Set}
	\subsubsection{Syntax}
		\texttt{set(arg0:option, arg1:option, ...)}
	
	\begin{phygdescription}
		{Changes the settings of \phyg. This command performs an array of functions
		from specifying the seed of the random number generator, to selecting a terminal for
		rooting output trees, to specifying  graph type, final assignment, and 	optimality 
		criterion. All \texttt{set} commands are executed at the start of a run, irrespective of
		where they appear in the script. The command \texttt{transform} is used to modify 
		global settings during a run (see \texttt{transform} Section \ref{subsec:Transform}).}
	\end{phygdescription}
			
	\subsubsection{Arguments}
		\begin{description}
			\item[compressResolutions:True|False] %\hl{change to BOOL} 
			Determines whether soft-wired graph 
			resolutions are ``compressed'' if multiple vertex assignments in alternate display 
			trees are equal in subtree leaf set, only the first lowest cost resolution is retained.
			This option can significantly reduce the time to evaluate softwired graphs, but can
			increase the optimality score of the graph.  Can be used in combination with
			\texttt{transform(compressResolutions:True)} or \texttt{transform(softwiredMethod:Exhaustive)} 
			at a later stage to improve graph score.
			
			\item[criterion:parsimony|pmdl] Sets the optimality criterion for graph search to be 
			method. Currently, parsimony and PMDL \citep{WheelerandVaron2022} are supported.
			
			\item[finalAssignment:DirectOptimization|DO|ImpliedAlignment|IA] Sets the method 
			of determining the ``final'' sequence states. DirectOptimization (DO) uses soft-wire the DO 
			method to assign the final states, which is more time consuming than \texttt{ImpliedAlignment}. 
			DO has an additional factor of potentially $O(n^2)$ in sequence length compared 
			to the constant factor for IA due to additional graph traversals.
			
			\item[graphFactor:nopenalty|W15|W23|PMDL] Sets the network penalty for a soft-wired network\\ 
			(\texttt{W15}),  (\texttt{W23}), or \texttt{PMDL} (for criterion = PMDL). W15 employs the
			parsimony network penalty of \cite{Wheeler2015}. W23 sets the parsimony network penalty akin to
			W15, but with the penalty applied to all blocks (more severe).
			
			\item[graphsSteepest:n] Sets the maximum number of graphs to be evaluated simultaneously during ``steepest'' descent 
			in swapping and network add/delete operations. The number is the minimum of the number of parallel threads and $n$
			(default 10).
			
			\item[graphType:tree|hardwired|softwired] Sets the phylogenetic graph type to tree, 
			hard-wired network, or soft-wired network. Forest are allowed by the network options.
			
			%\item[modelcomplexity] \hl{details}
			
			\item[outgroup:STRING] Specifies the terminal to root the output trees. 
			This name must appear in quotes. 
			
			\item[partitioncharacter:CHAR] Sets the character that is used to partition data. %\hl{The separator can be something else}
			
			\item[rootCost:noRootCost|W15|PMDL] Sets the root cost for a graph. W15 sets a 
			cost at $\frac{1}{2}$ the cost of `inserting' the root character assignments. 
			The W15 root cost is based on the same rationale as the parsimony network penalty of
			 \cite{Wheeler2015}.
			 
			 \item[seed:INTEGER] Sets the seed for the random number generator using the integer
			 value. If unspecified, \phyg uses the system time as the seed.
			 
			 \item[softwiredMethod:ResolutionCache|Exhaustive] Sets the algorithm for softwired graphs 
			 to the ``Exhaustive'' method of diagnosing all display trees as in \cite{Wheeler2015} or
			 the ``Resolution Cache'' method of \cite{WheelerandWashburn2023}.  
			 
		\end{description}
					
	\subsubsection{Defaults} 
		The default outgroup is the taxon whose name is 
		lexically first after renaming and taxon inclusion/exclusion. For this reason, it is best to specify 
		an outgroup explicitly. The default optimality criterion is \texttt{parsimony}, \texttt{CompressResolutions} 
		is set to \texttt{True}, \texttt{FinalAssignment} is set to \texttt{DirectOptimization}, and the default 
		graph type is \texttt{tree}. The default graphFactor is \texttt{W15} if parsimony is the optimality 
		criterion and \texttt{PMDL} if PMDL is set as the optimality criterion. The default rootCost
		is \texttt{noRootCost} if parsimony is the optimality criterion and \texttt{PMDL} if PMDL is set 
		as the optimality criterion.
		
	\begin{example}
		\item{\texttt{set(optimality:parsimony)}\\Sets the graph search optimality criterion to parsimony.}
						
		\item{\texttt{set(compressResolutions:False)}\\Turns off soft-wired graph resolution compression.}
	\end{example}

%---------------------------------------------
%support
%---------------------------------------------			
\subsection{Support}
	\subsubsection{Syntax}
		\texttt{swap(arg0, arg1:option, ...)}
		
	\begin{phygdescription}
		{Generates graphs supports via resampling \citep{Farrisetal1996} and Goodman-Bremer 
		\citep{Goodmanetal1982, bremer1994}. Currently, Jackknifing and Bootstrapping 
		are the resampling methods that are implemented.}
	\end{phygdescription}
		
	\subsubsection{Arguments}
		\begin{description}
			\item[buildonly] Performs very rapid, but not extensive graph searches for each 
			resampling replicate.
		
			\item[bootstrap] Calculates Bootstrap support. The user can specify the 
			number of iterations, using \texttt{replicates:n} (see below).
			When reported (via \texttt{report(support)}) edges are labeled with the bootstrap frequencies.
		
			\item[goodmanbremer|gb[:spr|tbr]] Specifies that Goodman-Bremer support is 
			calculated for input graphs. The method traverses the SPR or TBR neighborhood 
			as optionally specified (TBR as default) to determine an upper bound on the NP-hard 
			values (this is the method used in POY v2; \citealp{POY2} \textit{et seq.}). The number 
			of alternate graphs to be examined can be specified (\textit{ie.} limited) by \texttt{gbsample:[n]}. 
			When \texttt{gbsample:[n]} is specified, graphs are chosen uniformly at random.
		
			\item[jackknife:[n]] Specifies that Jackknife resampling is performed with $n$ acceptance 
			probability (default 0.6231 or $1 - e^{-1}$). When reported (via \texttt{report(support)}) 
			edges are labeled with the jackknife frequencies.
		
			\item[replicates:n] Specifies that $n$ resampling replicates (default 100) are performed 
			in resampling support for Jackknife and bootstrap methods.
		\end{description}	
		\subsubsection{Defaults}
			\texttt{support(goodmanBremer:TBR)}
		

		\begin{example}
			\item{\texttt{support(jackknife:0.50, replicates:1000)}\\Performs 1000 replicates of 
			delete 50\% jackknife resampling.}
				
			\item{\texttt{support(gb:SPR, gbSample:10000)}\\Produces Goodman-Bremer 
			support based on $10,000$ samples of the SPR neighborhood.}
		\end{example}

%---------------------------------------------
%swap
%---------------------------------------------		
\subsection{Swap} 
	\subsubsection{Syntax}
		\texttt{swap(arg0, arg1:option, ...)}
			
	\begin{phygdescription}
		{Performs branch-swapping rearrangement on graphs. This command implements a 
		group of algorithms referred to as branch swapping, that proceed by clipping
		parts of the given tree and attaching them in different positions. These algorithms 
		include ``NNI'' \citep{CaminandSokal1965, Robinson1971}, ``SPR'' \citep{Dayhoff1969}, 
		and ``TBR'' \citep{Farris1988, swofford1990a} refinement.}
	\end{phygdescription}
		
	\subsubsection{Arguments}
		\begin{description}
			\item[all]  Turns off all preference strategies to make a join, simply trying all possible 
			join positions for each pair of clades generated after a break, in a randomized order. 
			The refinement examines the entire rearrangement neighborhood of the current graph 
			before retaining the best (lowest cost) solutions.
		
			\item[annealing[:n]] Specifies that $n$ rounds of simulated annealing \citep{Metropolisetal1953, 
			Kirkpatricketal1983, Cerny1985} optimization are performed (default 1). The acceptance 
			of candidate graphs is determined by the probability $e ^ {- (c_c - c_b)/ (c_b * (k_{max} -k)/ k_{max})}$, 
			where $c_c$ is the cost of the candidate graph, $c_b$ is the cost of the current best graph, $k$ 
			is the step number, and $k_{max}$ is the maximum number of steps (set by the \texttt{steps:m}, 
			default 10) option.
		
			\item[drift[:n]] Specifies that $n$ rounds of the ``drifting'' form of simulated annealing 
			\citep{goloboff1999} optimization are performed (default 1) . The acceptance of candidate 
			graphs is determined by the probability $1/ (wf + c_c - c_b)$, where $c_c$ is the cost 
			of the candidate graph, $c_b$ is the cost of the current best graph, and $wf$ is the 
			\texttt{acceptWorse} (set by the \texttt{acceptWorse:m}, default 1.0) option. Equal 
			cost graphs are accepted with probability set by the \texttt{acceptEqual}  option. 
			\texttt{drift} differs from \texttt{annealing} in that there are no cooling steps to modify 
			acceptance probabilities. The maximum number of graphs changes is set by 
			\texttt{maxChanges}.
			
			\begin{description}
			
			\item[acceptEqual] 
			
			\item[acceptWorse] 
			
			\item[maxChanges:n] Specifies that drifting graph changes are limited to $n$ (default 15).
			
			\end{description}
		
			\item[ia] Specifies that Implied Alignment \citep{Wheeler2003} assignment are used for 
			branch swapping as opposed to full Direct Optimization for dynamic charters when the 
			graph type is \texttt{Tree}.
		
			\item[keep:n] Specifies that up to $n$ equally costly graphs are retained.
		
			\item[nni] Specifies that NNI refinement \citep{CaminandSokal1965, Robinson1971} is performed.
			
			\item[softwiredMethod:ResolutionCache|Exhaustive] Specifies that the algorithm for softwired graphs 
			is set to the ``Resolution Cache'' method of \cite{WheelerandWashburn2023} or ``exhaustive'' 
			method of diagnosing all display trees as in \cite{Wheeler2015}.
		
			\item[spr:[n]] Specifies that SPR refinement \citep{Dayhoff1969} is performed. If the optional 
			argument $n$ is specified, readdition of pruned graphs will be within $2 * N$ edges of its original 
			placement.
		
			\item[steps:n] Specifies that $n$ (default 10) temperature steps are performed during simulated 
			annealing (as specified by the \texttt{annealing}) option.
		
			\item[steepest] Specifies that refinement follows a greedy path, abandoning the neighborhood 
			of the current graph when a better (lower cost) graph is found.
		
			\item[tbr:[n]] Specifies that TBR refinement \citep{Farris1988, swofford1990a} is performed. If the 
			optional argument $n$ is specified, readdition of pruned graphs will be within $2 * N$ edges of its 
			original placement.
		\end{description}	
		
		\subsubsection{Defaults}
			\texttt{swap(spr, keep:1, steepest)}
		
		\begin{example}
			\item{\texttt{swap()}\\Performs spr branch swapping on each current graph returning the single 
			best rearrangement found for each graph employing steepest descent.}
			
			\item{\texttt{swap(tbr, all, keep:10)}\\Performs tbr branch swapping on each current graph 
			returning up to 10 best rearrangements found for each graph after examining all graphs in 
			the rearrangement neighborhood.}
		\end{example}
	
%---------------------------------------------
%transform
%---------------------------------------------		
\subsection{Transform}
	\label{subsec:Transform}
	\subsubsection{Syntax}
		\texttt{transform(arg0, arg1,...)}
			
	\begin{phygdescription}
		{\texttt{Transform} modifies global setting during program execution (as opposed to the \texttt{set} 
		command that operates at the inauguration of calculations). The command allows for changing 
		graph (e.g. Tree to Softwired Network) and data types (between dynamic and static approximation) 
		among other operations.}
	\end{phygdescription}
			
	\subsubsection{Arguments}
		\begin{description}
			\item[atRandom] In concert with \texttt{displayTrees:n} specifies that displays trees are chosen 
			uniformly at random for each input graph.
			
			\item[compressResolutions:True|False] %\hl{change to BOOL} 
			Determines whether soft-wired graph 
			resolutions are ``compressed'' if multiple vertex assignments in alternate display 
			trees are equal in subtree leaf set, only the first lowest cost resolution is retained.
			This option can significantly reduce the time to evaluate softwired graphs, but can
			increase the optimality score of the graph.  Can be used in combination with
			\texttt{transform(compressResolutions:True)} or \texttt{transform(softwiredMethod:Exhaustive)} 
			at a later stage to improve graph score.
			
			\item[displayTrees:[n]] When this option is specified, returns $n$ display trees for each graph 
			determined by the optional argument If the number of display trees is not 
			specified, 10 are returned. Used in concert with \texttt{toTree}.
			
			\item[dynamic] Reverts data type to the default ``dynamic'' for all dynamic homology 
			\citep{Wheeler2001} character types (e.g. DNA sequences). After this command, 
			graph optimization proceeds in the default manner with sequence characters treated 
			in their non-aligned (``dynamic'') condition.
			
			\item[dynamicEpsilon] Sets the level at which heuristic graph costs are verified by full (and time consuming) traversal.  
			Candidate graphs with costs within dynamicEpsilon of the current best graph are verified.
			
			\item[first] In concert with \texttt{displayTrees:n} specifies that the first $n$ displays tree 
			resolutions are chosen for each input graph.
			
			\item[outgroup:STRING]  Sets the outgroup to the taxon with the name STRING.
			
			\item[softwiredMethod:ResolutionCache|Exhaustive] Specifies that the algorithm for softwired graphs 
			is set to the ``Resolution Cache'' method of \cite{WheelerandWashburn2023} or ``exhaustive'' 
			method of diagnosing all display trees as in \cite{Wheeler2015}.
			
			\item[staticApprox] Converts non-aligned (``dynamic'') sequence characters to their Implied 
			Alignment \citep{Wheeler2003, WashburnandWheeler2020} condition.
			
			\item[toHardwired] Converts exiting graphs to hardwired network graphs.
			
			\item[toSoftwired] Converts exiting graphs to softwired network graphs.
			
			\item[toTree] Converts exiting graphs to trees. For both Softwired and Hardwired graphs 
			this proceeds via graph resolution of network nodes into ``display'' trees. Since there are up to 
			$2^n$ display trees for a graph with $n$ network nodes, this number can be quite large. 
			The number of display trees produced for each graph is controlled via the options 
			\texttt{displayTrees:n}, \texttt{atRandom}, and \texttt{first}. 
		\end{description}
			
		\subsubsection{Defaults}
			None.
		
		\begin{example}
			\item{\texttt{transform(toSoftwired)}\\Converts each current graph to a softwired network graph.}
					
			\item{\texttt{transform(staticApprox)}\\Changes data to all static characters via Implied Alignment 
			for further analysis.}
					
		\end{example}
	


%\chapter{Program Usage}
%\section{Example Script Files}
%Tutorials are available for download on the \phyg \href{https://github.com/amnh/PhyGraph}{GitHub}
%website. These tutorials provide guidance for using the program. Each tutorial contains a \phyg script 
%that includes detailed commentaries explaining the rationale behind each step of the analysis. The 
%command arguments will differ substantially depending on type, complexity, and size of the data set. 
%
%%	The following file (titled ``Example Script 1'') reads two input sequence files (net-I.fas and net-II.fas), 
%%	skips all 	the lines that begin with double dash (\texttt{--}), reads the graph file net-I-II.dot, sets the 
%%	outgroup to the taxon named ``zero,'' specifies the graph type for the analysis is a softwired network, 
%%	and 	reports a series of files with various information about the data and graphs.
%%	
%%		\begin{verbatim}
%%			-- Example Script 1
%%			read("net-I.fas")
%%			--read("net-Ia.fas")
%%			--read("net-IIa.fas")
%%			read("net-II.fas")
%%			--read("net-I.dot")
%%			--read("net-I.tre")
%%			--read("net-II.tre")
%%			--read("net-II.dot")
%%			read("net-I-II.dot")
%%			set(outgroup:"zero")
%%			set(graphtype:softwired)
%%			report("net-test.tre", graphs, newick, overwrite)
%%			report("net-test.dot", graphs, dot, overwrite)
%%			report("net-test-data.csv", data, overwrite)
%%			report("net-test-diag.csv", diagnosis, overwrite)
%%			report("net-display.dot", displaytrees, dot, overwrite)
%%			report("net-display.tre", displaytrees, newick, overwrite)
%%		\end{verbatim}
%	
%\section{Faster and Slower}
%Multiple options affect both the quality of results (better ot worse optimality score) and
%the overall execution time.  In general, the more time consuming options also have larger
%memory footprints.
%
%\subsection{Evaluation of Graphs}
%	\texttt{Multitraverse}\\
%	\texttt{CompressResolutions}\\
%	\texttt{SoftwiredMethod}\\
%
%\subsection{Search Options}
%	\texttt{steepest}\\
%	\texttt{Limiting number of network edges}\\
%	\texttt{Limiting number of graphs}
%	
%\section{Parallel Evaluation}
%	\texttt{+RTS -NX -RTS}
%	
%\section{Memory Use}
%
%The amount of memory used during program execution can be reported by adding the 
%runtime option ``-s'' as in \texttt{+RTS -s -RTS} to the command line (runtime options can 
%be specified together as in \texttt{+RTS -NX -s -RTS}). This will output several fields of 
%data with the  ``in use'' field specifying the maximum amount of memory requested from 
%the OS. RTS options are described in detail at \url{https://downloads.haskell.org/ghc/latest/docs/users_guide/runtime_control.html}. 
%	
	% Optimize memory consumption--keep low number of graphs in initial searches, later keep 
% a larger number to get others 
%		\hl{(see email from WW 04-21-22)}
%		\hl{where should I discuss this?}


\section*{Acknowledgments}
	The authors would like to thank DARPA SIMPLEX N66001-15-C-4039, the Robert J. 
	Kleberg Jr. and Helen C. Kleberg foundation grant ``Mechanistic Analyses of Pancreatic 
	Cancer Evolution'', and the American Museum of Natural History for financial support. 
	
	%\newpage
	%\bibliography{big-refs-3.bib}
	\bibliography{/Users/louise/DropboxAMNH/big-refs-3.bib}
	%\bibliography{/home/ward/Dropbox/Work_stuff/manus/big-refs-3.bib}
	%\bibliography{/users/ward/Dropbox/work_stuff/manus/big-refs-3.bib}
 \end{document}