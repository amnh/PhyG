\documentclass[11pt]{article}
\usepackage{longtable}
\usepackage{color}
\usepackage{tabu}
\usepackage{setspace}
\usepackage{pdflscape}
\usepackage{graphicx}
%\usepackage{subfigure}
\usepackage{caption}
\usepackage{subcaption}
\usepackage{natbib}
\usepackage{fullpage}
%\bibliographystyle{plain}
\bibliographystyle{cbe}
\usepackage{algorithmic}
\usepackage[vlined,ruled]{algorithm2e}
\usepackage{amsmath}
\usepackage{amsfonts}
\usepackage{amssymb}
\usepackage{url}
\usepackage[dvipsnames]{xcolor}
\usepackage[colorlinks=true, linkcolor=blue, citecolor=DarkOrchid, urlcolor=TealBlue ]{hyperref}
\usepackage[nottoc,numbib]{tocbibind}

\let \oriAlgorithm=\algorithm%
\renewcommand{\algorithm}[2]{%
	\oriAlgorithm%
	\SetAlgoLined%
	\DontPrintSemicolon%
	\SetKwComment{Comment}{$\triangleright\ $}{}%
	\TitleOfAlgo{ #1 }%
	\caption{ #2 }%
}%
\newcommand*{\Set}[2]{ #1 $\gets$ #2 \;}
\newcommand*{\SetC}[3]{ #1 $\gets$ #2 \Comment*[r]{#3}}
\newcommand*{\ForC}[3]{\For(\Comment*[f]{#2}){#1}{#3}}

\begin{document}
	%\firstpage{1}
	
	\title{PhylogeneticGraph\\Program Documentation\\Version 0.1}
	%\subtitle{Program Documentation\\Version 0.1}
	\author{Ward C. Wheeler\\
		Division of Invertebrate Zoology,\\ American Museum of Natural History,\\ 200 Central Park West, New York, NY, 10024, USA;\\wheeler@amnh.org}
	
	
	\maketitle
%	\begin{center}
%		Running Title: {PhyGraph Version 0.1}
%	\end{center}
	\newpage
	
	\tableofcontents
	\newpage
	\section{Introduction}
	This document contains descriptions of algorithms, procedures, data structures and other aspects of the 
	source code documentation for the program PhylogeneticGraph (\texttt{PhyG}).  
	
	 \texttt{PhyG} is a sucessor program to \texttt{POY} \citep{POY2,POY3,POY4,Varonetal2010,POY5, Wheeleretal2015} \url{https://github.com/wardwheeler/POY5}, 
	 although a ``complete'' Haskell rewrite, optimized C (and even some assembler) was ported over from POY for pairwise alignment of small alphabet (<8)
	 sequences.  These functions are access via the Haskell FFI.
	
	
	
	\section{Overview of Code base}
	Source code structure.
	
	
		
	\section {Command Parsing}
	
	\section{Post-Order Graph Traversal}
		\subsection{Trees}
		A decorated Graph (tree) is created for each character for each block for the graph.  For exact characters, where no addition traversals are required, 
		the specified or default
		outgroup sets the direction of the graph.  For non-exact (e.g. sequence) characters the best traversal rooting is stored for each character in each block
		although the cost of the graph is recalculated based on the best traversal (over all edges in the graph), the preliminary (post-order) states are not propagated back
		to the decorated graph (third field of phylogenetic graph).  After the pre-order pass, the final states are propagated back.
		Vertices are not renumbered during the rerooting process, so indices remain unchanged.
		
			Preliminary states (post-order) are determined for exact and non-exact characters as in \cite{Wheeler2012}.
	
	\section{Pre-Order Graph Traversal}
		\subsection{Trees}
		Final state assignments of root vertices are set to the preliminary, post-order state.
		Final  states are propagated back to the decorated graph (third field of phylogenetic graph).
		Vertices are not renumbered during the rerooting process, so indices remain unchanged.
		
		Final states (pre-order) are determined for exact and non-exact characters as in \cite{Wheeler2012}.  Currently final states for 
		non-exact characters (e.g. sequence) are set as the median between the gapped preliminary state of the vertex and the final state of its parent
		(for a tree), `extra' gaps in preliminary state are propagated to the gaped left and right descendant sequences, left, right, and parent final sequences
		should now line up and a 3-median can be calculated.
		

	\section{Adding New Character Types}
		Current character types include Additive, Non-Additive, Matrix, Slim Sequences, Wide Sequences, and Huge Sequences.  
		Functions that branch on character types need to be updates and are found in:
		\begin{itemize}
			 \item{GraphOptimization.Medians.hs}
			 	\begin{itemize}
			 		\item{Median2Single}
					\item{Median2SingleStaticIA}
					\item{Union2Single}
					\item{GetPrealignedUnion}
					\item{getPreAligned2Median}
			 	\end{itemize}
			 \item{GraphOptimization.PreOrderFunctions.hs}
			 	\begin{itemize}
			 		\item{updateCharacter}
			 		\item{getCharacterDistFinal}
			 		\item{setFinal}
			 	\end{itemize}
			 \item{Commands.Transform.hs}
				\begin{itemize}
				\item{transformCharacter}
			\end{itemize}
			 \item{Commands.CommandExecution.hs}
				\begin{itemize}
				\item{}
			\end{itemize}
			 \item{Types.Types.hs}
			 \begin{itemize}
			 	\item{}
			 \end{itemize}
			 \item{Utilities.Utilities.hs}
				\begin{itemize}
				\item{}
				\end{itemize}
			 \item{Utilities.ThreeWayfunctions.hs}
				\begin{itemize}
				\item{threeMedianFinal}
			\end{itemize}
			 \item{Support.Support.hs}
				\begin{itemize}
				\item{}
			\end{itemize}
			 \item{Functions with ``== NonAdd'' etc will need extra cases for any new character type}
	\end{itemize}
	\subsection{Execution in Parallel}
	By default the program will execute using a single process core.  By specifying the options `+RTS -NX -RTS' where `X' is the number of processors offered to the program. These are specified after the program as in (for 4 parallel threads):\\
	\\
	PhyGraph +RTS -N4 -RTS other options...  \\
	
	Parallel code options are set using a parmap-type strategy throughout the code.  The basic definitions of this functionality are found in ParallelUtilities.hs
	
	\section*{Acknowledgments}
	The author would like to thank DARPA SIMPLEX N66001-15-C-4039, the  Robert J. Kleberg Jr. and Helen C. Kleberg foundation grant ``Mechanistic Analyses of Pancreatic Cancer Evolution'', and the American Museum of Natural History for financial support.  
	
	\newpage
	%\bibliography{big-refs-3.bib}
	%\bibliography{/users/ward/Dropbox/Work_stuff/manus/big-refs-3.bib}
	\bibliography{/home/ward/Dropbox/Work_stuff/manus/big-refs-3.bib}
\end{document}