\documentclass[]{article}
\usepackage{longtable}
\usepackage{color}
\usepackage{tabu}
\usepackage{setspace}
\usepackage{pdflscape}
\usepackage{graphicx}
\usepackage {float}
%\usepackage{subfigure}
\usepackage{caption}
\usepackage{subcaption}
\usepackage{natbib}
%\usepackage{fullpage}
\bibliographystyle{plain}
\usepackage{fancyhdr}
%\bibliographystyle{cbe}
\usepackage{algorithmic}
\usepackage[vlined,ruled]{algorithm2e}
\usepackage{amsmath}
\usepackage{amsfonts}
\usepackage{amssymb}
\usepackage[T1]{fontenc}
\usepackage{url}
 
\usepackage[dvipsnames]{xcolor}
\usepackage{color, soul}
\usepackage[colorlinks=true, linkcolor=blue, citecolor=DarkOrchid, urlcolor=TealBlue ]{hyperref}
%\usepackage[nottoc,numbib]{tocbibind}
\usepackage{tocloft}


\setlength\itemindent{0.25cm}

\newcommand{\phyg}{\texttt{PhyG} }
\newcommand{\BigO}[1]{\ensuremath{\mathcal{O}\left(\,#1\,\right)}\xspace}

\title{PhyG 0.2 Tutorials}

\author{Louise M. Crowley}
\makeindex
\begin{document}
\maketitle

\pagestyle{fancy}

%Maybe add:\\
%\begin{enumerate}
%	\item{reports of diagnosis}
%	\item{Supports--Goodman-Bremer, bootstrap, jackknife--maybe on chel or 
%	something preassigned and small}
%	\item{Consensus etc outputs}
%\end{enumerate}

\section{\phyg Tutorial 2}

These tutorials are intended to provide guidance for using the phylogenetic 
program \texttt{PhyG}. Each tutorial contains a \phyg script that includes 
detailed commentaries explaining the rationale behind each step of the analysis. 
The command arguments will differ substantially depending on type, complexity, 
and size of the data set. The values of arguments within this tutorial have been 
chosen such that the analysis can complete within the timeframe of this session.
Therefore, the values used here should not be taken to be optimal parameters. 

The tutorials use sample datasets that can also be downloaded from the \texttt{PhyG} 
\href{https://github.com/amnh/PhyGraph}{GitHub} website. The minimally required 
items to run the tutorial analyses are the \phyg application and sample data files. 
A complete guide to the commands and associated arguments of \phyg can be found 
\href{https://github.com/amnh/PhyGraph}{here}.

%-------------------------------------------------------------------------------------------------------
\subsection{Running a timed search with \phyg}
\label{subsec:timedsearch}

In this tutorial, you will learn how to run a timed search with \phyg .  

%\printindex

\end{document}
